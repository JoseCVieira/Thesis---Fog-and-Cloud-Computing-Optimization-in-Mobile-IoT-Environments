%!TEX encoding = UTF-8 Unicode
\section{Introduction}\label{sec:Introduction}
%Context - CLOUD COMPUTING
\subsection{Context}
\noindent
Cloud computing became popular at the beginning of the twenty-first century. The \textit{National Institute of Standards and Technology} (NIST) defines cloud computing as a model for enabling ubiquitous, convenient, on-demand network access to a shared pool of configurable computing resources (e.g., networks, servers, storage, applications, and services) that can be rapidly provisioned and released with minimal management effort or service provider interaction \cite{mell2011nist}. Cloud computing has been instrumental in expanding the reach and capabilities of computing, storage, and networking infrastructure to the applications. In this model, clients outsource the allocation and management of resources (hardware or software) that they rely upon to external entities known as clouds. Clouds provide consumers online accesses to computing services and centralized data storage, running on data centers where large groups of servers, disks, and routers are networked. Since the demand for cloud resources change over time, setting a fixed amount of resources results in either over-provisioning or under-provisioning, so \textit{cloud service providers} (CSPs) afford dynamic resources for a scalable workload, applying a pay-as-you-go cost model where clients only pay for the amount of resources they actually use.\\
%CSPs offer three main types of services, namely infrastructure, platform, and software as services (IaaS, PaaS, SaaS), in descending order of flexibility, which users may choose depending on their application needs. IaaS allows cloud customers to directly accesses IT infrastructure for processing, storage, networking, over the Internet. Clients can configure the IaaS (often offered as a standalone VM), in terms of hardware (e.g., number of CPU cores, RAM capacity) and corresponding software for his need. PaaS offer users a framework they can build upon to develop or customize applications. It manages the underlying low-level processes and allows users to focus on managing software for its applications. Finally, SaaS is the most popular way of using cloud computing. It utilizes the internet to deliver applications to its users, which are managed by a third-party vendor. This service is helpful if the user likes to get full software packages, and do not want to take care of software issues, such as database scalability, socket management, etc.\\
%Cloud computing provides customers with yet another feature that they may use taking into account both security and configurability levels that their applications require, which includes private cloud, community cloud, public cloud, and hybrid cloud.\\
IoT apps...
The prosperity of Internet of Things (IoT) and the
success of rich Cloud services have expedited the emergence of
a new computing paradigm called Fog computing, which promotes
the processing of data at the proximity of their sources.
Complementary to the Cloud, Fog promises to offer many appealing
features, such as low latency, low cost, high multitenancy, high
scalability, and to consolidate the IoT ecosystem.

\noindent\tab Although cloud computing has bring forth many advantages, the time required to access cloud-based services may not be suitable for some applications with ultra-low latency requirements. Also, the rapid growth in the number of connected IoT devices has brought new needs, such as greater demand for high-bandwidth, geographically-dispersed, low-latency, and privacy-sensitive data processing. These demands require cloud resources to be closer to end-devices, making plenty of paradigms such as \textit{Fog Computing} (FC) to emerge. %Each of them focus on different environments with different characteristics, being FC the most general one.\\
%Among those paradigms are \textit{Fog Computing} (FC), \textit{Mobile Computing} (MC), \textit{Mobile Cloud Computing} (MCC), \textit{Mobile Ad hoc Cloud Computing} (MACC), \textit{Edge Computing} (EC),\textit{ Multi-access Edge Computing} (MEC), \textit{Cloudlet Computing} (CC), \textit{Mist Computing} (mist) [ref].\\[6pt]

%Context - FOG COMPUTING
\noindent\tab Fog computing is a new computing architecture that aims to enable computing, storage, networking, and data management not only in the cloud, but also along the IoT-to-Cloud path as data traverses to the cloud (preferably close to the IoT devices).
%OpenFog Consortium defines fog computing as a horizontal system-level architecture that distributes computing, storage, control and networking functions closer to the users along a cloud-to-thing continuum \textbf{[ref]}
Although fog computing is intended to provide strong support for the Internet of Things, it does not replace the needs of cloud-based services. In fact, fog and cloud complement each other; one cannot replace the need of the other. By coupling cloud and fog computing, the services can be optimized even further, allowing enhanced capabilities for data aggregation, processing, and storage. Fog nodes can be placed close to IoT source nodes, due to low hardware footprint and low power consumption (e.g., small servers, routers, switches, gateways, set-top boxes, access points), allowing latency to be much smaller compared to traditional cloud computing. Also, the decentralized nature of fog computing allows devices to either serve as fog nodes themselves or use fog resources as a client ?????. Moreover, Internet connectivity is not essential for the fog-based services to work, what means that services can work independently and send necessary updates to the cloud whenever the connection is available \cite{yousefpour2018all}.


MOBILE devices are ubiquitous in people’s everyday
life, with a remarkable growth of mobile data traffic
over recent years [2]. As mobile applications become increasingly
resource-hungry, the gap between required resources.
To bridge this gap, cloud computing can be used to expand
mobile devices resources. To deal with high latency of distant
cloud center, the concept of cloudlet was introduced in [3]
where it is defined as a trusted, resource-rich computer or
cluster of computers well-connected to the Internet and available
for use by nearby mobile devices. A cloudlet represents
a container for virtual machines (VMs): connected users
are associated with VMs supporting low-latency application
offloading use-cases.
Cloudlet concept is expected to be supported by 3-tier
hierarchical network provisioning as presented in [4] and [5].
In this hierarchy the cloudlet is the primal resource for the
augmentation of the mobile device capabilities, while a remote
cloud is used as last available resource, or for delay-tolerant
resource-intensive applications.

[2] “CVN index: Global mobile data traffic forecast update, 2015–2020,”
Cisco, San Jose, CA, USA, White Paper no. c11-520862, 2016.
[3] M. Satyanarayanan, P. Bahl, R. Caceres, and N. Davies, “The case for
VM-based cloudlets in mobile computing,” IEEE Pervasive Comput.,
vol. 8, no. 4, pp. 14–23, Oct./Dec. 2009.
[4] Y. Jararweh et al., “Resource efficient mobile computing using cloudlet
infrastructure,” in Proc. IEEE MSN, Dec. 2013, pp. 373–377.
[5] (Dec. 2016). Elijah. [Online]. Available: http://elijah.cs.cmu.edu




Here,
it is envisioned that entities (such as basestations in a cellular network) closer to the network edge would host smallersized
cloud-like infrastructure distributed across the network. This idea has been variously termed as Cloudlets [1], Fog Computing [2], Edge Computing [3], and Follow Me Cloud [4], to name a few. The trend towards edge-clouds is expected
to accelerate as more users perform a majority of their computations on handhelds and as newer mobile applications get
adopted.
[1] M. Satyanarayanan, Cloudlets: At the leading edge of cloud-mobile convergence, in: Proceedings of the 9th International ACM Sigsoft Conference on
Quality of Software Architectures, Ser. QoSA’13, ACM, New York, NY, USA, 2013, pp. 1–2.
[2] F. Bonomi, R. Milito, J. Zhu, S. Addepalli, Fog computing and its role in the Internet of things, in: Proceedings of the First Edition of the MCC Workshop
on Mobile Cloud Computing, ACM, 2012, pp. 13–16.
[3] S. Davy, J. Famaey, J. Serrat-Fernandez, J. Gorricho, A. Miron, M. Dramitinos, P. Neves, S. Latre, E. Goshen, Challenges to support edge-as-a-service, IEEE
Commun. Mag. 52 (1) (2014) 132–139.
[4] T. Taleb, A. Ksentini, Follow me cloud: interworking federated clouds and distributed mobile networks, IEEE Netw. 27 (5) (2013) 12–19.


Rather than applying
the Fog concept to a specific area, this paper is focused on
the realization of Fog.

One of the key design issues in edge-clouds is service migration: should a service currently running in one of the edge-clouds
be migrated as the user locations change, and if yes, where? This question stems from the basic tradeoff between the
cost of service migration vs. the reduction in network overhead and latency for users that can be achieved after migration.
While conceptually simple, it is challenging to make this decision in an optimal manner because of the uncertainty in user
mobility and request patterns. Because edge-clouds are distributed at the edge of the network, their performance is closely
related to user dynamics. These decisions get even more complicated when the number of users and applications is large
and there is heterogeneity across edge-clouds. Note that the service migration decisions affect workload scheduling as well
(and vice versa), so that in principle these decisions must be made jointly.
[231 - all one needs to know...]


Augmented
reality applications that use head-tracked systems, for example,
require end-to-end latencies to be less than 16 ms [1]. Cloudbased
virtual desktop applications require end-to-end latency
below 60 ms if they are to match QoS of local execution [2].
Remotely rendered video conference, on the other hand,
demand end-to-end latency below 150 ms [3]. Limited battery
life, computation and storage capacity constraints inherent
to mobile devices mean that application executions must be
offloaded to cloud servers, which then return processed results
to the mobile devices through the Internet. When cloud servers
reside in remote data centers, end-to-end communication may
translate into long delays characteristic of multi-hops transmissions
over the Internet. Moving cloud computing to the edge of
a network has helped to lessen these otherwise unacceptable
delays while leveraging the benefits of a high-performance
cloud, e.g., Fog Computing [4], Cloudlets, [5] and Follow Me
Cloud [6]. While this improvement is not trivial, delivering
cloud services from the edge of the network is not, by itself,
sufficient to meet latency QoS. When a mobile client initially
secures a one-hop away edge cloud server to ensure the shortest
network delay, client mobility may cause the server to be
be multi-hops away. The increased network distance, and the
potential bottleneck bandwidth that might be introduced by the
intermediate links may result in poor connectivity to the cloud
service. Even when a mobile user moves around the originally
connected edge cloud, service latency may increase because of
unexpected crowds of mobile clients seeking to connect to the
same edge cloud simultaneously. Thus, increased network or
server processing delays may violate acceptable latency QoS
constraints.
Cloud service migration may effectively provide expected
QoS with respect to user mobility, dynamic networks and
varying edge cloud states. To date, previous CloudNet [7]
and VM Handoff [8] studies introduced virtual machine (VM)
migration in real time under the assumption that the allimportant
variables of when and where to migrate were known.
These assumptions cannot be made in the real world for
two reasons. First of all, conditions that may or may not
trigger migration of a cloud service may vary widely. One
central consideration that must be accounted for is the tradeoff
between the cost of migration and any real QoS improvement.
Secondly, we need to quantify long-term performance of cloud
servers with respect to any requested service migration to
ensure that the best server is chosen, wherein that best choice
is realized by the maximization of promised QoS for any
mobile user over time.
Previous work [9], [10] proposed a static distance-based
MDP model that solved the problem of where to migrate
by defining each edge cloud migration possibility according
to hop counts between it and a mobile user. A cost/reward
function in MDP may be used to measure the trade-off
between migration costs and performance gains, and it may
be used to calibrate the long-term performance improvements
by each edge server with respect to any prospective mobile
client. Therefore, these studies did work to prove the feasibility
of applying MDP with respect to the where to migrate
decision. But the static distance-based MDP models did not
fully support real-time mobile applications due to its inherent
limitation to reflect the two ruling factors in any where to
migration decision: 1)network state, and 2) server state.
The inherent limitation of previous distance-based models
to reflect the two ruling factors in any where to migration decision
can be illustrated in the fairly common instance of two
edge clouds deemed identical because they both share an equal
hop count to a mobile user. Yet, no two edge clouds are ever
precisely identical because they are characterized by different
network delays and different server processing delays. These
differences provide a real difference in the choice between
one edge cloud and another of equal hop count. For example,
one edge cloud may become heavily loaded and congested
because it has the smallest hop count for the client, and,
accordingly, be the chosen destination for a specific migration.
Another unaddressed problem exists, because a previous MDP
edge cloud service migration model recalculates optimal edge
cloud migration for mobile users without specifying an optimal
interval period for such recalculation. Since running MDP
is a computing intensive task, short recalculation intervals
introduce the heavy overhead to the server. Conversely, longer
recalculation intervals may translate into lazy migration resulting
in periods of transgression of QoS guarantees. Finally, of
course, operating MDP for edge cloud migration requires realtime
feeds of pertinent parameters into the MDP model, while
previous MDP models assume the parameters as static. Such
assumptions make these models impractical, if not impossible,
to apply to dynamic applications, network states or server
states in the real world of real time.


With the advance of wireless technologies, the rapid
mobile data traffic growth will lead to severe network resource
consumption and exceptionally long latency to access services,
especially for cloud-based applications.



Simply applying existing radio access-oriented
MM (mobility management) schemes leads to poor performance mainly due to the
co-provisioning of radio access and computing services of the
MEC-enabled (mobile edge computing) BSs (base stations).


Cloud computing is an emerging computing technology that uses internet to maintain large
applications and data servers to provide services to end users belonging to different organization. It has make
availability of resources in a secure and flexible manner thereby enhancing the throughput and performance .It
has certain limitation that the load balancing of data and hosting of cloud data centers in the internet create large
and unpredictable network latency. It is then resolved by new sort of computing model called Fog Computing.
Fog is analogous to cloud only difference is that it is located at the edge of network. Another advantage is that
the applications which requires location dependence can be make feasible through it..

Cloud computing is a newest computing paradigm which make resources like applications and files available
through internet having thousands of computers interlinked together in a distributed and complex manner and
you have to pay for the resources as you use them .Cloud Computing uses the concepts of virtualization,
distributed computing, networking, software and web computing for its implementation. A cloud consists of
several elements such as clients, datacenter and distributed server. The focus in cloud computing is to reduce
execution time of tasks on the machines which are known as virtual machines that run in parallel with each
other. Cloud Computing provide different service models according to different user needs like Infrastructure as
a service known as IAAS, Platform as a service known as PAAS, Software as a service known as SAAS
providing a cost effective manner to pay for services on utility costing basis. In this multiple users are able to
access more than one server without purchasing and considering the updating of license for different software
applications. So it is basically SOA (service oriented architecture). It offer many advantages like fault tolerance,
high availability, reduced overhead, flexibility, scalability, reliability, location independence, elimination of
system administrative functions.


Limitations of Cloud Computing:
Though cloud computing offer many advantages but it has certain limitations too providing high speed reliable
internet connectivity, high latency, multi-homing to avoid link outages ,requirement of high capacity bandwidth,
security and privacy as data has to travel a long path from server to users though if we encrypt data. The
emerging trend of Internet of things[10] that want every device or objects to be connected on the Internet,
sensor network , real time applications, mobile data applications that require fast transfer of data and bandwidth
put limitation on cloud computing. Real time applications which require high speed streaming of data when user
directly interact with them suffer from delay jitter caused by latency in network which need to be solved by
cloud computing.
These shortcomings can be resolved by the introduction of Fog Computing which provides better quality of
service in terms of reduced data traffic, low latency, location awareness, low bandwidth, mobility which is made
feasible by making fog computing systems very close to end users.
The other difference between fog computing and cloud computing is that in former the distance between client
and server require one hop in while in later it requires multiple hops as the signal get weak while passing
through various servers, also cloud computing required leased line while fog computing is entirely based on
wireless network. Despite of many differences both Fog and Cloud complement each other as Cloud Computing
will be preferred for high end batch processing jobs and Fog Computing will be preferred for more resilient and
providing more security to the emerging technology.


Introduction to Fog Computing:
Fog Computing is a term which is introduced by the Cisco systems, it is basically extended or new model of
cloud computing in which we will connect enormous wireless data objects in distributed environment by placing
the data and resources at the edge of cloud rather than hosting and working from a centralized cloud. In this
bandwidth can be reduced as there is no need for transferring every bit of information over cloud network rather
we can aggregate data at some access points which lower costs , increases efficiency, throughput ,reduced data
traffic ,lower power delay also provide security as data transfer has not to follow a long path.
Fog is nothing but a virtualized platform like cloud only difference lies in the fact it is closer to the ground.
Hence Fog Computing is called as an edge computing because Fog System operate from network ends. In
this decision are taken as close to the place where data is generated as much as possible and stop it sending to
reach Global network which is done initially by training fraction of data on machine learning models after the
results are found accurate then this model is implemented in to the devices.


Fog computing is a new computing architecture, composed of a set of
near-user edge devices called fog nodes, which collaborate together in
order to perform computational services such as running applications,
storing an important amount of data, and transmitting messages. Fog
computing extends cloud computing by deploying digital resources at
the premise of mobile users. In this new paradigm, management and
operating functions, such as job scheduling aim at providing high-performance,
cost-effective services requested by mobile users and executed
by fog nodes.


%The problem
\subsection{The problem}
\noindent\tab Despite the benefits of using fog computing, the current model suffer from several problems. \cite{Armbrust:10} \\

\subsection{Motivation}
The motivation of this paper is to fill the gap by proposing
a layered Fog framework to better support IoT applications,
encompassing all the layers along the Cloud-to-Things continuum
through virtualization. In particular, the virtualization
refers to the creation of hardware, operating system, storage
device, network resource and event processing by abstraction,
orchestration and isolation. In our context, the virtualization is
further divided into object virtualization [17], network function
virtualization [18], and service virtualization [19].

[17] C. Sarkar et al., “DIAT: A scalable distributed architecture for IoT,”
IEEE Internet Things J., vol. 2, no. 3, pp. 230–239, Jun. 2015.
[18] R. Mijumbi et al., “Network function virtualization: State-of-the-art
and research challenges,” IEEE Commun. Surveys Tuts., vol. 18, no. 1,
pp. 236–262, 1st Quart., 2015.
[19] H. Ko, J. Jin, and S. L. Keoh, “Secure service virtualization in IoT
by dynamic service dependency verification,” IEEE Internet Things J.,
vol. 3, no. 6, pp. 1006–1014, Dec. 2016.

For example, heterogeneous sensory
nodes (sensors, controllers, actuators, etc.) on a driverless car
are estimated to generate about 1 GB data per second [3]

A. D. Angelica. Google’s Self-Driving Car Gathers Nearly 1 GB/Sec.
Accessed: Dec. 6, 2016. [Online]. Available: http://www.kurzweilai.net/
googles-self-driving-car-gathers-nearly-1-gbsec


The concept of “Fog” was initially proposed in [5], where
its role was defined as an extension of Cloud to support some
appealing features, such as low latency, heterogeneity, and
mobility. More specifically, Fog might be specified in terms
of functionality as Fog edge nodes (FENs), Fog server (FS),
and Foglet, where FENs and FS are hardware nodes, and
Foglet is the middleware in charge of data exchange, as
presented in Fig. 1 [6].

[5] F. Bonomi, R. Milito, J. Zhu, and S. Addepalli, “Fog computing and its
role in the Internet of Things,” in Proc. MCC Workshop Mobile Cloud
Comput., Helsinki, Finland, Aug. 2012, pp. 13–16.
[6] J. Li, J. Jin, D. Yuan, M. Palaniswami, and K. Moessner, “EHOPES:
Data-centered Fog platform for smart living,” in Proc. Telecommun.
Netw. Appl. Conf. (ITNAC), Sydney, NSW, Australia, Nov. 2015,
pp. 308–313.


Starting from available simulators a significant programming
effort is required to obtain a simulation tool meeting the actual
needs. (deve estar no trabalho a realizar, não na intro..)

keyworks:

the main reasons lie on Fig. 1
Observe that
As a matter of fact,
In a nutshell,
As we shall show,
aforementioned benefits
continuum
We tackle the edge cloud network design problem
which is, to the best of our knowledge, an untreated problem in the literature
The trend towards edge-clouds is expected
Meanwhile, in order
have been considered as means to provide computing capabilities
Furthermore, 



\subsection{Alternatives}
\noindent\tab [Alternatives]\\

%The problem
\subsection{Our approach}
\noindent\tab To address the aforementioned problems, the present document proposes ...\\

%Contributions
\subsection{Contributions}
\noindent\tab [Contributions]\\
%Road map
\noindent\tab The remainder of the document is structured as follows. Section \ref{sec:Goals} xxx. Section \ref{sec:RelatedWork} xxx. Section \ref{sec:Architecture} describes xxxx. Section \ref{sec:Evaluation} defines the xxx. Finally, Section \ref{sec:Schedule} presents xxxx and Section \ref{sec:Conclusion} xxxx.


