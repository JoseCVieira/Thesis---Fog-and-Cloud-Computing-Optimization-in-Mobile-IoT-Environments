%!TEX encoding = UTF-8 Unicode
\section{Introduction}\label{sec:Introduction}
%Context - CLOUD COMPUTING
\subsection{Context}
\noindent
Cloud computing became popular at the beginning of the twenty-first century. The \textit{National Institute of Standards and Technology} (NIST) defines cloud computing as a model for enabling ubiquitous, convenient, on-demand network access to a shared pool of configurable computing resources (e.g., networks, servers, storage, applications, and services) that can be rapidly provisioned and released with minimal management effort or service provider interaction \cite{mell2011nist}. Cloud computing has been instrumental in expanding the reach and capabilities of computing, storage, and networking infrastructure to the applications. In this model, clients outsource the allocation and management of resources (hardware or software) that they rely upon to external entities known as clouds. Clouds provide consumers online accesses to computing services and centralized data storage, running on data centers where large groups of servers, disks, and routers are networked. Since the demand for cloud resources change over time, setting a fixed amount of resources results in either over-provisioning or under-provisioning, so \textit{cloud service providers} (CSPs) afford dynamic resources for a scalable workload, applying a pay-as-you-go cost model where clients only pay for the amount of resources they actually use.\\
%CSPs offer three main types of services, namely infrastructure, platform, and software as services (IaaS, PaaS, SaaS), in descending order of flexibility, which users may choose depending on their application needs. IaaS allows cloud customers to directly accesses IT infrastructure for processing, storage, networking, over the Internet. Clients can configure the IaaS (often offered as a standalone VM), in terms of hardware (e.g., number of CPU cores, RAM capacity) and corresponding software for his need. PaaS offer users a framework they can build upon to develop or customize applications. It manages the underlying low-level processes and allows users to focus on managing software for its applications. Finally, SaaS is the most popular way of using cloud computing. It utilizes the internet to deliver applications to its users, which are managed by a third-party vendor. This service is helpful if the user likes to get full software packages, and do not want to take care of software issues, such as database scalability, socket management, etc.\\
%Cloud computing provides customers with yet another feature that they may use taking into account both security and configurability levels that their applications require, which includes private cloud, community cloud, public cloud, and hybrid cloud.\\
\noindent\tab Although cloud computing has bring forth many advantages, the time required to access cloud-based services may not be suitable for some applications with ultra-low latency requirements. Also, the rapid growth in the number of connected IoT devices has brought new needs, such as greater demand for high-bandwidth, geographically-dispersed, low-latency, and privacy-sensitive data processing. These demands require cloud resources to be closer to end-devices, making plenty of paradigms such as \textit{Fog Computing} (FC) to emerge. %Each of them focus on different environments with different characteristics, being FC the most general one.\\
%Among those paradigms are \textit{Fog Computing} (FC), \textit{Mobile Computing} (MC), \textit{Mobile Cloud Computing} (MCC), \textit{Mobile Ad hoc Cloud Computing} (MACC), \textit{Edge Computing} (EC),\textit{ Multi-access Edge Computing} (MEC), \textit{Cloudlet Computing} (CC), \textit{Mist Computing} (mist) [ref].\\[6pt]

%Context - FOG COMPUTING
\noindent\tab Fog computing is a new computing architecture that aims to enable computing, storage, networking, and data management not only in the cloud, but also along the IoT-to-Cloud path as data traverses to the cloud (preferably close to the IoT devices).
%OpenFog Consortium defines fog computing as a horizontal system-level architecture that distributes computing, storage, control and networking functions closer to the users along a cloud-to-thing continuum \textbf{[ref]}
Although fog computing is intended to provide strong support for the Internet of Things, it does not replace the needs of cloud-based services. In fact, fog and cloud complement each other; one cannot replace the need of the other. By coupling cloud and fog computing, the services can be optimized even further, allowing enhanced capabilities for data aggregation, processing, and storage. Fog nodes can be placed close to IoT source nodes, due to low hardware footprint and low power consumption (e.g., small servers, routers, switches, gateways, set-top boxes, access points), allowing latency to be much smaller compared to traditional cloud computing. Also, the decentralized nature of fog computing allows devices to either serve as fog nodes themselves or use fog resources as a client ?????. Moreover, Internet connectivity is not essential for the fog-based services to work, what means that services can work independently and send necessary updates to the cloud whenever the connection is available \cite{yousefpour2018all}.

%The problem
\subsection{The problem}
\noindent\tab Despite the benefits of using fog computing, the current model suffer from several problems. \cite{Armbrust:10} \\

\subsection{Alternatives}
\noindent\tab [Alternatives]\\

%The problem
\subsection{Our approach}
\noindent\tab To address the aforementioned problems, the present document proposes ...\\

%Contributions
\subsection{Contributions}
\noindent\tab [Contributions]\\
%Road map
\noindent\tab The remainder of the document is structured as follows. Section \ref{sec:Goals} xxx. Section \ref{sec:RelatedWork} xxx. Section \ref{sec:Architecture} describes xxxx. Section \ref{sec:Evaluation} defines the xxx. Finally, Section \ref{sec:Schedule} presents xxxx and Section \ref{sec:Conclusion} xxxx.


