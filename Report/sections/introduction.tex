%!TEX encoding = UTF-8 Unicode
\chapter{Introduction}\label{sec:Introduction}
%Context - IoT
World is growing at a fast pace and so is data. Agility and flexibility of big data applications are gradually taking the form of the Internet of Things (IoT), comprising \textit{things} that have unique identities and are connected to the Internet, being accessible from anywhere in the world. Ubiquitous deployment of interconnected devices is estimated to reach 50 billion units by 2020 \cite{evans2011internet}. This exponential growth is broadly supported by the increasing number of mobile devices (e.g., smart phones, tablets), smart sensors (e.g., autonomous transportation, industrial controls, wearables), wireless sensors and actuators networks. The number of mobile devices are predicted to reach 11.6 billion by 2021, exceeding the world’s projected population at that time (7.8 billion), where the subset of wearable ones are expected to be 929 million \cite{CiscoVis16:online}.\\
\noindent\tab Managing the data generated by IoT sensors and actuators is one of the biggest challenges faced when deploying an IoT system. Although this kind of devices has evolved radically in the last years, battery life, computation and storage capacity remain limited. This means that they are not suitable for running heavy applications, being necessary, in this case, to resort to third parties.\\
%Context - CLOUD COMPUTING
\noindent\tab Cloud Computing (CC) is a resource-rich environment that has been imperative in expanding the reach and capabilities of IoT devices. It enables clients to outsource the allocation and management of resources (hardware or software) that they rely upon to the cloud. In addition, to avoid over- or under-provisioning, Cloud Service Providers (CSPs) also afford dynamic resources for a scalable workload, applying a pay-as-you-go cost model. This way, cloud computing is the on-demand delivery of compute power, database storage, applications, and other IT resources through a cloud services platform via the Internet with pay-as-you-go pricing \cite{WhatisCl79:online}. Besides, it also brings other advantages such as availability, flexibility, scalability, reliability, to mention a few.\\
\noindent\tab Despite the benefits of cloud computing, there are two main problems, linked to IoT applications, which remain unresolved. The first and the most obvious, is the fact that cloud servers reside in remote data centers. Consequently, the end-to-end communication may be subject to long delays (characteristic of multi-hops transmissions over the Internet). Some applications, with ultra-low latency requirements, can't support such delays. Augmented reality applications which use head-tracked systems, for example, require end-to-end latencies to be less than 16 ms \cite{ellis2004generalizeability}. Cloud-based virtual desktop applications require end-to-end latency below 60 ms if they are to match Quality of Service (QoS) of local execution \cite{taylor2015virtual}. Remotely rendered video conference, on the other hand, demand end-to-end latency below 150 ms \cite{szigeti2005end}. On the other hand, the exponential growing number of IoT devices raises the second problem: as the number of connected devices increases, the bandwidth required to support them becomes too large for centralized processing (i.e. CC). To overcome these drawbacks, there are immediately two apparent solutions: (1) to increase the number of centralized cloud data centers, which will be too costly, and (2) to get more efficient with the data sent to the cloud.\\
\noindent\tab The solution which has already been proposed is to bring the cloud closer to the end devices, where entities such as base-stations would host smaller sized clouds. This idea has brought the emergence of several computing paradigms such as cloudlets \cite{satyanarayanan2013cloudlets}, fog computing \cite{bonomi2012fog}, edge computing \cite{davy2014challenges}, and follow me cloud \cite{taleb2013follow}, to name a few. These are different solutions, often confused in the literature, which provide faster approaches and gain better situational awareness in a more timely manner. Regardless of their characteristics, they all share the same goal, implementing solution (2).\\
%Context - FOG COMPUTING
\noindent\tab As it will be discussed later, fog computing, also known as fog networking or fogging, is the most comprehensive and natural paradigm to get more efficient with the data sent to the cloud. A simple definition of fog is ``cloud closer to the ground'', which gives an idea of its functioning. Fog is thus a decentralized computing infrastructure that aims to enable computing, storage, networking, and data management not only in the cloud, but also along the cloud-to-thing path as data traverses the network towards the cloud. Essentially, it extends cloud computing and services along the network itself, bringing them closer to where data is created and acted upon. Fog brought these services closer to the end devices due to its low hardware footprint and low power consumption. This way, both problems raised by the use of cloud computing, may be solved, or at least significantly mitigated. First, the path travelled by the data in the sense-process-actuate model is much shorter (ideally, just one hop to send data and another to receive the results), allowing latency to be much smaller compared to the traditional cloud computing. Second, through geographical distribution, there are significant amounts of data that are no longer travelling up to the cloud. Fog computing, prefers to process data, as much as possible, in the nodes closer to the edge of the network. In a simplistic manner, it only considers transferring the data further if there is not enough computational power to meet the demands.\\
\noindent\tab Nevertheless, cloud is still more suitable than fog for massive data processing, when the latency constraints are not so tight. Therefore, even though fog computing has been proposed to grant support for IoT applications, it does not replace the needs of cloud-based services. In fact, fog and cloud complement each other. Together, they offer services even further optimized to IoT applications. It should be noted that, Internet connectivity is not essential for the fog-based services to work, which means that services can work independently and send necessary updates to the cloud whenever the connection is available \cite{yousefpour2018all}.\\
\noindent\tab Nonetheless, it is worth mentioning that similarly to cloud computing, fog can use the concept of virtualization to grant heterogeneity. IoT applications may span many Operating Systems (OSs) and application environments (e.g., Android, iOS, Linux, Windows), as well as diverse approaches to partitioning and offloading computation. There is churn in this space from new OS versions, patches to existing OS versions, new libraries, new language run-time systems, and so on. In order for fog to support all these variants, it can be introduced a level of abstraction that cleanly encapsulates this messy complexity in a Virtual Machine (VM) or a container. Moreover, it enables applications to coexist in a physical server (host) to share resources. Meanwhile, in fog computing, the use of virtualization is also crucial to provide mobility. As the end devices become more distant from their current connected fog server, their application needs to be migrated to a more favourable spot in order to be able to met its QoS requirements (e.g., latency) and improve its Quality of Experience (QoE).\\
\noindent\tab Summing up, in this context, virtualization is a vital technology at different levels namely: (1) isolation between untrusted user-level computations, (2) mechanisms for authentication and access control, (3) dynamic resource allocation for user-level computations, (4) ability to support a wide range of user-level computations, with minimal restrictions on their process structure, programming languages or OSs, (5) mobility, migration and tasks offloading mechanisms, (6) power efficiency, and (7) fault tolerance.\\
\section{Motivation}\label{subsec:Motivation}
%The problem / Motivation
Despite the benefits that fog promises to offer, such as low latency, heterogeneity, scalability and mobility, the current model suffers from some limitations that must be overcome.\\
%The problem / Motivation - MOBILE FOG COMPUTING
\noindent\tab There is lack of support for mobile fog computing. Most of the existing literature assumes that the fog nodes are fixed, or only considers the mobility of IoT devices \cite{yousefpour2018all}. Less attention has been paid to mobile fog computing and how it can improve the QoS, cost, and energy consumption. For instance, a bus or a train could have computational power; as a fog node, it could provide offloading support to both end devices (inside and outside it) and other fog servers. The same could be applied to cars that are nowadays getting increasingly better in terms of computational power. Both would be extremely useful to enhance the resources and capabilities of fog computing. Specially in environments such as large urban areas, where traffic congestion is frequent or when those are parked (e.g., while an electric vehicle is charging). On top of that, it would reduce the implementation costs since it would no longer require such computational power in the fixed fog nodes. Finally, it would reduce the costs to the client both in terms of latency and energy consumption, since the fog nodes to which they are connected may be even closer.\\
%The problem / Motivation - MULTI-OBJECTIVE FOG SYSTEM DESIGN
\noindent\tab Another limitation of fog computing is to take into account few parameters in the decision-making of migration. Most of the existing schemes that are proposed for fog systems, such as offloading, load balancing, or service provisioning, only consider few objectives (e.g., QoS, cost) and assume other objectives do not affect the problem \cite{yousefpour2018all}. Fog servers are less powerful than clouds due to the high deployment cost. If many requests are made to the same fog node at the same time, it will not have enough computational and storage capabilities to give a prompt response. So, it raises the question: \textit{should a service currently running in one fog node be migrated to another one, and if yes, where?} While conceptually simple, it is challenging to make these decisions in an optimal manner. Offloading tasks to the next server (i.e. upstream server) seems to be the solution, however, to migrate either the VM or the container that was initially one-hop away from the IoT device to a multi-hop away server, will increase the network distance. Consequently, it raises the end-to-end latency and the bandwidth usage by the intermediate links. Besides, this decision still has to take into account the cost for both the client (e.g., migration time, computational delay) and the provider (e.g., computing and migration energy). Ignoring some of these variables can lead to wrong decisions that will both violate latency constraints of users' applications and damage or defeat the credibility of fog computing.
%Alternatives ?? \\
%Our approach ?? \\
%Contributions ?? \\

\section{Objectives}\label{subsec:Objectives}
This work intends to tackle two of the current limitations that are little or no treated in the literature. One is to provide mobility support in fog computing environments, not exclusively to the end devices but also to the fog nodes, and the other is to achieve multi-objective fog system design. These objectives shall be implemented in a toolkit allowing the simulation of resource management techniques in IoT and mobile fog computing environments. In order to achieve the aforementioned goals, this work involves studying the current mobility approaches that are publicly available, with respect to IoT or fog nodes. Also, analyze several optimization algorithms adopted in the field of data placement, to propose a novel architecture and to select the toolkit in which the proposed solution will be implemented and evaluated.

\section{Outline}\label{subsec:Outline}
The remainder of the document is structured as follows. Section \ref{sec:RelatedWork} presents a background section followed by the state-of-the-art and a review of relevant works in the context of the objectives described above. Section \ref{sec:Architecture} describes our proposal to achieve those objectives. Section \ref{sec:Evaluation} defines the methodology of how the results obtained will be evaluated. Finally, Section \ref{sec:Schedule} presents the scheduling of the future work and Section \ref{sec:Conclusion} concludes the document.