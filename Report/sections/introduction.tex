%!TEX encoding = UTF-8 Unicode
\section{Introduction}\label{sec:Introduction}

%Context
\noindent
Cloud computing became popular at the beginning of the twenty-first century. This model has been instrumental in expanding the reach and capabilities of computing, storage, and networking infrastructure to the applications. According to the National Institute of Standards and Technology (NIST), cloud computing is a model for enabling ubiquitous, convenient, on-demand network access to a shared pool of configurable computing resources (e.g., networks, servers, storage, applications, and services) that can be rapidly provisioned and released with minimal management effort or service provider interaction \textbf{[ref]}. In this model, clients outsource the allocation and management of resources (hardware or software) that they rely upon to external entities known as clouds. Clouds provide consumers online accesses to computing services and centralized data storage, typically running on data centers where large groups of servers, disks, and routers are networked. Those resources can be dynamically reconfigured for a scalable workload, therefore cloud services that are offered with a pay-as-you-go cost model, only charge to users for the amount of resources they use.\\


Cloud offers three main types of services namely infrastructure, platform, and software as services (IaaS, PaaS, SaaS), in descending order of flexibility offered, which application developers can use depending on the needs of the applications they develop. 



With IaaS, customer accesses the resources, IT infrastructure such as storage, processing, networking, over the Internet. The client can configure the IaaS (often offered as a standalone VM), in terms of hardware and corresponding software for his need, namely, the number of CPU cores and RAM capacity, in addition to systems-level software. PaaS, offer users a framework they can build upon to develop or customize applications, and also fully support software lifecycle - often based on a middleware - for software management and configuration. PaaS manages the underlying low-level processes and allows users to focus on managing software for its applications. Finally, SaaS is the most popular way of using cloud computing. SaaS utilizes the internet to deliver applications to its users, which are managed by a third-party vendor. It is helpful if the user likes to get full software packages, and do not want to take care of software issues, such as database scalability, socket management, etc. Since the demand for cloud resources is not fixed and can change over time, setting a fixed amount of resources results in either over-provisioning or underprovisioning[ref]. A foundation of cloud computing is based on provisioning only the required resources for the demand. Since it is difficult to estimate service usage from tenants, most cloud providers have a pay-as-you-go payment scheme. As a result, providers can be more flexible on how to provision resources, and clients only pay for the amount of resources they actually use. Cloud computing provides customers with yet another feature that clients may use taking into account both security and configurability levels that their applications require, which includes private cloud, community cloud, public cloud, and hybrid cloud. Although cloud computing has bring forth many advantages to IT operations, the time required to access cloud-based applications may be too high and may not be suitable for some applications with ultra-low latency requirements. Also, the rapid growth in the number of connected IoT devices has brought new needs, such as greater demand for high-bandwidth, geographically-dispersed, low-latency, and privacy-sensitive data processing. These demands require cloud resources to be closer to where the data is generated, emerging new paradigms such as edge computing and fog computing.

Fog computing is a new computing architecture that aims to enable computing, storage, networking, and data management not only in the cloud, but also along the IoT-to-Cloud path as data traverses to the cloud (preferably close to the IoT devices). Fog computing is defined by the OpenFog Consortium [6] as "a horizontal systemlevel architecture that distributes computing, storage, control and networking functions closer to the users along a cloud-to-thing continuum." Although fog computing is intended to provide strong support for the Internet of Things, it does not replace the needs of cloud-based services. In fact, fog and cloud complement each other; one cannot replace the need of the other. By coupling cloud and fog computing, the services can be optimized even further, allowing enhanced capabilities for data aggregation, processing, and storage. Fog nodes can be placed close to IoT source nodes, due to low hardware footprint (e.g., utilizes small servers, routers, switches, gateways, set-top boxes, access points), allowing latency to be much smaller compared to traditional cloud computing. Fog computing provides moderate availability of computing resources at lower power consumption [23]. Also, the decentralized nature of fog computing allows devices to either serve as fog nodes themselves or use fog resources as a client. As might be expected, fog computing can be accessed through connected devices from the edge of the network to the network core and does Internet connectivity is not essential for the fog-based services to work, what means that services can work independently with no Internet connectivity and send necessary updates to the cloud whenever the connection is available. Many industries could use fog to their benefit: energy, manufacturing, transportation, healthcare, smart cities, to mention a few.

-- Another paradigms --
Another paradigms have emerged to meet the needs that require cloud resources to be closer to end devices namely mobile computing, mobile cloud computing, mobile ad hoc cloud computing, edge computing, multi-access edge computing, cloudlet computing, mist computing, etc[ref]. These paradigms nao sao stadards entao...

\cite{Armbrust:10}

%The problem
xxxx

%Alternatives
xxx

%Our approach
To address the aforementioned problems, the present document proposes ...

%Contributions

%Road map
The remainder of the document is structured as follows. Section \ref{sec:Goals} xxx. Section \ref{sec:RelatedWork} xxx. Section \ref{sec:Architecture} describes xxxx. Section \ref{sec:Evaluation} defines the xxx. Finally, Section \ref{sec:Schedule} presents xxxx and Section \ref{sec:Conclusion} xxxx.


