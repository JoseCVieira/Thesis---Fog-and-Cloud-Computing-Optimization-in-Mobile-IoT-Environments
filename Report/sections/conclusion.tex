%!TEX encoding = UTF-8 Unicode
\section{Conclusion}
\label{sec:Conclusion}
%Fazer o resumo do trabalho efectuado, retomando a idea, as contribuicoes definidas e a forma como estas se materializam.
With the evolution of technology, new challenges raise. Regarding IoT devices, the main faced difficulties are lack of processing, memory and battery capabilities. These are even more pronounced when applications are computationally demanding. Unfortunately, cloud computing is not able to solve them. One can deploy applications to a remote cloud. However, the experienced latency will be too costly due to multi-hop communication. Therefore, the processing time reduction will not pay the price of latency for applications with defined time boundaries constraints. Fog computing raises to fill this gap. It brings cloud closer to the \textit{things}, reducing the end-to-end latency to a negligible value. Still, fog is a new computing paradigm and has some challenges to be resolved. This work intends to tackle, firstly, the lack of mobility support for mobile fog nodes and mobile users (as discussed, there is no support for both in the literature).
Secondly, once server and network state are not static (the experienced end-to-end latency may vary), with regard to migration support, this work aims to provide multi-objective decision-making optimization. Finally, in order to allow future researchers to further improve and validate fog computing, it is also objective to implement the defined architecture, in Section \ref{sec:Architecture}, in an open-source simulation toolkit.
