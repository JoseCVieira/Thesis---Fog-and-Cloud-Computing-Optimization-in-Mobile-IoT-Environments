%!TEX encoding = UTF-8 Unicode
\section{Related Work}
\label{sec:RelatedWork}
% Show what you read Start by presenting the structure of component. Show each
% item in a different section. Which works are known as relevant in this area
% Each section should end with a comparative evaluation. End each chapter with a
% synthesis, a table about various solutions, which features are interesting,
% which we want. Finnish sections with a work summary on a single sentence.

The solution proposed in this document leverages knowledge obtained from studying several concepts and systems from the current state-of-the-art. In this section, an overview of those concepts and systems will be given, stating for each of them their advantages and disadvantages. This section is structured as follows. Section \ref{sec:Computingparadigms} presents different methods to push intelligence and computing power closer to the source of the data and why this work adopted fog computing for this purpose. Section \ref{sec:fog_architecture} describes the generic fog computing architecture, its actors and the different orchestration approaches. Section \ref{sec:Migration} discusses several optimization algorithms regarding migration of virtual resources (e.g., VM). Finally, Section \ref{sec:Toolkits} shows several open-source simulators of fog-related computing paradigms along with their characteristics.

%!TEX encoding = UTF-8 Unicode
\subsection{Related Computing Paradigms}
\label{sec:Computingparadigms}
In what concerns fog computing standardization, there is a lack of unanimity. As aforementioned, fog has been variously termed as cloudlets, edge computing, etc. Different research teams are proposing many independent definitions of fog (and fog-related computing paradigms). As there is a research gap in the definitions and standards for fog computing, this work follows the definitions that Ashkan Yousefpour et al. \cite{yousefpour2018all} present. Below, are described some paradigms that were raised in order to bring cloud closer to the end devices, as well as their pros and cons. As a conclusion, we show why fog computing is the natural platform for IoT.

\subsubsection{Mobile Computing}\label{subsec:MC}
Mobile/Nomadic Computing (MC) is characterized by the processing being performed by mobile devices (e.g., laptops, tablets, or mobile phones). It is necessary to overcome the inherent limitations of environments where connectivity is sparse or intermittent and where there is low computing power. As this model only uses mobile devices to provide services to clients, there is no need for extra hardware. They already have communication modules such as Bluetooth, WiFi, ZigBee, etc. As already mentioned, mobile devices have evolved in recent years. However, their resources are more restricted, compared to fog and cloud. This computing paradigm has the advantage of being characterized by a distributed architecture. The disadvantages of MC are mainly due to their hardware nature (i.e. low resources, balancing between autonomy and the dependency of other mobile devices; characteristic that prevails in all distributed architectures) and the need for mobile clients to efficiently adapt to changing environments \cite{satyanarayanan1996fundamental}. MC alone may not be able to meet the requirements of some applications. On the one hand, it is limited due to autonomy constraints and, on the order hand by low computational and storage capacity. This restricts the applications where this paradigm is feasible. For instance, it is unsuitable for applications that require low-latency and which, at the same time, generate large amounts of data that needs to be stored or processed. Nonetheless, MC can use both fog and cloud computing to enhance its capacities, not being restricted to a local network, expanding the scope of mobile computing and the number of applications where it can be used.

\subsubsection{Mobile Cloud Computing}
Cloud and fog computing, as mentioned in Section \ref{subsec:MC}, are key elements for validate the importance of MC. This interaction between them results in a new paradigm, called mobile cloud computing (MCC). MCC, differs from MC in the sense that mobile applications can be partitioned at runtime, so that computationally intensive components of the application can be handled through adaptive offloading \cite{shiraz2013review}, from mobile devices to the cloud. This characteristic increases the autonomy of mobile devices (i.e battery lifetime), as both the data storage and data processing may occur outside of them. Also, it enables a much broader range of mobile subscribers, rather than the previous laptops, tablets, or mobile phones. Unlike the resource-constrained MC, MCC has high availability of computing resources, scaling the type of applications where it is possible to use (e.g., augmented reality applications). Unlike MC, MCC relies on cloud-based services, where its access is done through the network core by WAN connectivity, which means that applications running on these platforms require connection to the Internet all the time. On the one hand, both MCC and MC suffer from the intrinsic characteristics of mobility, such as frequent variations of network conditions (intensified under rapid mobility patterns), and, on the other hand, even if the mobile devices remain fixed, MCC suffers from the inherent disadvantage of using cloud-based services (i.e. communication latency), which makes it unsuitable for some delay-sensitive applications with heavy processing and high data rate demands.

\subsubsection{Mobile Ad hoc Cloud Computing}
In some scenarios, there exists lack of infrastructure or a centralized cloud, so to implement a network based on MCC may not always be suitable. To overcome dependency on an infrastructure, Mobile Ad hoc Cloud Computing (MACC) was proposed \cite{hubaux2001toward}. It consists of a set of mobile nodes that form a dynamic and temporary network enabled by routing and transport protocols. These nodes consist of mobile ad hoc devices, which may continuously join or leave the network. In order to counteract the aforementioned characteristics inherent to this type of networks, and unlike MC, a set of ad hoc devices may form a local cloud that can be used in the network for purposes of storage and computation. Mobile Ad hoc NETworks (MANET), are imperative in use cases such as disaster recovery, car-to-car communication, factory floor automation, unmanned vehicular systems, etc. Although MANETs do not rely on external cloud-based services as MCC does, which mitigates the latency problem, it shares some limitations inherent to MC and ad hoc networks such as the power consumption constraints. Moreover, the formed local cloud may still be computationally weak and, as both network and cloud are dynamic, it is more challenging to achieve an optimal performance (i.e. as there is no infrastructure, mobile devices are also responsible for routing traffic among themselves).

\subsubsection{Edge Computing}
Edge Computing (EC), makes use of connected devices at the edge of the network to enhance its capabilities (i.e. management, storage, and processing power). It is located in the local IoT network, being ideally located one hop away from the IoT device and at most located a few hops away. Open Edge Computing defines EC as a computation paradigm that provides small data centers (edge nodes) in proximity to the users, enabling a dramatic improvement in customer experience through low latency interaction with compute and storage resources just one hop away from the user \cite{OpenEdge73:online}. As the connected devices don't have to wait for a centralized platform to provide the requested service, nor are so limited in terms of resources as in the traditional MC, their service availability is relatively high. Also, the restrictions over the autonomy are not so tight once there are not only mobile devices. Nonetheless, EC has some drawbacks. Latency, in this context, is composed by three components: data transmission time, processing time and result receiving time. Even though the communication latency is negligible, processing time may be significant. This computing paradigm only uses edge devices, whose computation and storage capabilities may still be poor (e.g., routers, switches), compared to fog or cloud computing, so this processing latency may still be too high for some applications.\\
\noindent\tab OpenFog Consortium states that fog computing is often erroneously called edge computing, but there are key differences between the two concepts \cite{OpenFog0208}. Although they have similar concepts, edge computing tends to be limited to the edge devices (i.e. located in the IoT node network), excluding the cloud from its architecture. Whereas, fog computing is hierarchical and it is not limited to a local network, but instead it provides services anywhere from cloud to \textit{things}. It is worth noting that the term edge used by the telecommunication's industry usually refers to 4G/5G base stations, Radio Access Networks (RANs), and Internet Service Provider (ISP) access/edge networks. Yet, the term edge that is recently used in the IoT landscape refers to the local network where sensors and IoT devices are located \cite{yousefpour2018all}.

\subsubsection{Multi-access Edge Computing}
Analogously, MCC is an extension of MC through CC, as Multi-access Edge Computing (MEC) is an extension of MC through EC (telecommunication's industry definition). ETSI defines MEC as computation paradigm that offers application developers and content providers cloud-computing capabilities and an IT service environment at the edge of the network. This environment is characterized by ultra-low latency and high bandwidth as well as real-time access to the radio network, that can be leveraged by applications \cite{ETSIMult81:online}. In MEC, operators can open their RAN edge to authorized third parties, allowing them to deploy applications and services towards mobile subscribers through 4G/5G base stations. The first approach to the edge of a network meant the edge of a mobile network, hence the name Mobile Edge Computing. As MEC research progressed, it was noticed that the term leaves out several access points that may also construct the edge of a network. This prompted the change from Mobile Edge Computing to Multi-access Edge Computing in order to reflect that the edge is not solely based on mobile networks \cite{MobileEd74:online}. Now it includes a broader range of applications beyond mobile device-specific tasks, such as video analytics, connected vehicles, health monitoring, augmented reality, etc. Similar to EC, MEC can operate with little to no Internet connectivity and use small-scale data centers with virtualization capacity to provide services. MEC is expected to benefit significantly from the up-and-coming 5G platform, as it allows for lower latency and higher bandwidth among mobile devices, and it supports a wide range of mobile devices with finer granularity.\\
\noindent\tab Both fog computing and MEC have the objective of offering similar type of features. Fog computing concentrates on applications, mainly IoT, that take advantage of a platform set that collectively assist end devices. MEC, on the other hand, focuses on application-related enhancements in terms of feedback mechanisms, information and content processing and storage, etc. \cite{taleb2017multi}.

\subsubsection{Cloudlet Computing}
Cloudlet Computing (cC) is another direction in mobile computing that aims to bring cloud closer to end devices through the use of cloudlets. M. Satyanarayanan et al. states that a cloudlet is a trusted, resource-rich computer or cluster of computers that is well-connected to the Internet and available for use by nearby mobile devices \cite{satyanarayanan2009case}. Cloudlet is, as the name suggests, a smaller sized cloud with lower computational capacity. It can be seen as a ``data center in a box'', where mobile users can exploit their VM to rapidly instantiate customized-service software in a thin client fashion. This way, it is possible to offload computation from mobile devices to VM-based cloudlets located on the network edge (telecommunication industry definition). Through those VMs, cloudlets are capable of providing resources to end devices in real-time over a WLAN network. The relatively low hardware footprint, results in moderate computing resources, but lower latency and energy consumption and higher bandwidth compared to cloud computing. The characteristics of this computing paradigm make it possible to handle applications with low-latency requirements, supporting real-time IoT applications. Y. Jararweh et al. \cite{jararweh2013resource} propose an architecture mobile-cloudlet-cloud, where they present three reasons which indicate that even though cloudlets are computationally powerful, they still need a connection to the cloud and its services: (1) heavy non real time jobs might be processed in the enterprise cloud while the real-time ones would be processed by the cloudlet, (2) accessing a file stored in the enterprise cloud, and (3) accessing some services that are not available inside the cloudlet. Although cloudlet computing fits well with the mobile-cloudlet-cloud architecture, fog computing offers a more generic alternative that natively supports large amounts of traffic, and allows resources to be anywhere along the cloud-to-things continuum. As it will be shown later, cloudlets are great resources and, in this way, they can be combined with the fog computing paradigm.

%muita coisa copiada, tentar reescrever
\subsubsection{Mist Computing}
Mist computing emerges to push IoT analytics to the ``extreme edge''. This computing paradigm is an even more dispersed version of fog. That means locating analytics tools not just in the core and edge, but also at the ``extreme edge'' \cite{Ciscopus95:online}. Mist computing layer is composed by mist nodes that are perceived as lightweight fog nodes. They are more specialized and dedicated nodes with low computational resources (e.g., microcomputers, microcontrollers) that are even closer to the end devices than the fog nodes \cite{iorga2018fog}. Therefore, mist computing can be seen as the first (non-mandatory) layer in the IoT-fog-cloud continuum. It extends compute, storage, and networking across the fog through the \textit{things}. This decreases latency and increases subsystems' autonomy. It can be implemented in order to enhance the services of predominance of wireless access and mobility support. The challenge with implementing mist computing systems lies in the complexity and interactions of the resulting network. These must be managed by the devices themselves as central management of such systems is not feasible.

\subsubsection{Concluding Remarks}
As was already mentioned, there are some other similar computing paradigms such as Follow Me Cloud (FMC), Follow Me Edge (FME), Follow Me edge-Cloud (FMeC) and Cloud of Things (CoT), to name a few. However, this state-of-the-art section had as first objective to investigate the most concepts addressed in the literature. The purpose was to understand their characteristics and to identify current limitations that must be tackled by novel solutions, in order to allow the deployment of delay-sensitive IoT systems in mobile environments. Table \ref{computing_paradigms} compares the features of the paradigms described above.\\
%Fig. \ref{computing_paradigms} shows a classification of the paradigms described above, as well as how they overlap in terms of scope.\\
\noindent\tab These computing paradigms present different pros and cons, having been proposed to cover different use cases. Even so, fog computing is suited for many use cases, including data-driven computing and low-latency applications, being the most versatile and comprehensive one. As aforementioned, fog is flexible enough to interact and take advantage of other paradigms such as edge, cloud, cloudlet and mist computing. Nonetheless, it may not be suitable for a few extreme use cases, such as disaster recovery or sparse network topologies where ad hoc computing (e.g., MACC) may be a better fit.
%\begin{figure}[t]
%	\centering
%	\includegraphics[width=80mm]{images/computing_paradigms}
%	\caption{A classification of scope of fog computing and its related computing paradigms (adapted from \cite{yousefpour2018all}).}
%	\label{computing_paradigms}
%\end{figure}
\begin{table}[!t]
	\scriptsize
	\begin{tabular*}{\textwidth}{l >{\centering\arraybackslash}m{0.4in} >{\centering\arraybackslash}m{0.4in} >{\centering\arraybackslash}m{0.4in} >{\centering\arraybackslash}m{0.4in} >{\centering\arraybackslash}m{0.4in} >{\centering\arraybackslash}m{0.4in} >{\centering\arraybackslash}m{0.4in} >{\centering\arraybackslash}m{0.4in} >{\centering\arraybackslash}m{0.4in}}
		\toprule
		\centering\textbf{Feature} & \textbf{CC} & \textbf{MC} & \textbf{FC} & \textbf{EC} & \textbf{MCC} & \textbf{MACC} & \textbf{MEC} & \textbf{cC} & \textbf{mist} \\[2pt]
		\toprule
		Heterogeneity support & \cmark &  & \cmark & \cmark & \cmark &  &  &  & \cmark \\ \midrule
		Infrastructure need & \cmark &  & \cmark & \cmark & \cmark &  & \cmark & \cmark & \cmark \\ \midrule
		Geographically distributed &  &  & \cmark & \cmark &  &  & \cmark & \cmark & \cmark \\ \midrule
		Location awareness &  & \cmark & \cmark & \cmark &  & \cmark & \cmark & \cmark & \cmark \\ \midrule
		Ultra-low latency &  &  & \cmark & \cmark &  &  & \cmark & \cmark & \cmark \\ \midrule
		Mobility support &  & \cmark & \cmark & \cmark & \cmark & \cmark & \cmark & \cmark & \cmark \\ \midrule
		Real-time application support &  &  & \cmark & \cmark &  &  & \cmark & \cmark & \cmark \\ \midrule
		Large-scale application support & \cmark &  & \cmark & \cmark &  &  & \cmark &  & \cmark \\ \midrule
		Standardized & \cmark & \cmark & \cmark & \cmark &  &  & \cmark &  &  \\ \midrule
		Multiple IoT Applications & \cmark &  & \cmark &  &  &  &  & \cmark & \cmark \\ \midrule
		Virtualization support & \cmark &  & \cmark &  &  &  & \cmark & \cmark &  \\ \bottomrule \\
	\end{tabular*}
	\caption{Features of fog computing related paradigms (adapted from \cite{yousefpour2018all}).}
	\label{computing_paradigms}
	\vspace{-5mm}
\end{table}
%!TEX encoding = UTF-8 Unicode
\subsection{Fog Computing Architecture}
\label{sec:fog_architecture}
\noindent Fog computing is a great resource to support IoT applications' requirements in mobile environments. Taking into account what has been mentioned in Section \ref{sec:Introduction} and Section \ref{sec:Computingparadigms}, it has the following fundamental characteristics which validate the statement uttered above (refer to Table \ref{computing_paradigms}):
\begin{itemize}
	\item \textbf{Heterogeneity support}. Supports collection and processing of data of different actors acquired through multiple types of network communication, wide diversity applications and services;
	\item \textbf{Geographical distribution}. Uses anything between the cloud and \textit{things} to provide ubiquitous computing, allowing continuity of service in mobile environments;
	\item \textbf{Contextual location awareness, and low latency}. Provides low latency due to the proximity between the IoT devices and the fog nodes. Also, the contextual location allows them to be aware of the cost of communication latency with both other fog nodes and the end devices, allowing the distribution of applications across the network to be organized in a weighted manner;
	\item \textbf{Mobility support}. The exponential growth of mobile devices demands support for mobility techniques, such Locator/ID Separation Protocol (LISP), that decouples the device identity from its location, requiring a distributed directory system;
	\item \textbf{Real-time interactions}. Applications may involve real-time interactions rather than batch processing (e.g., as cloud does);
	\item \textbf{Scalability and agility of federated, fog-node clusters}. Fog is adaptive; may form clusters-of-nodes or cluster-of-clusters to support elastic compute, resource pooling, etc., supporting large-scale applications;
	\item \textbf{Multiple IoT applications}. Fog devices handle multiple IoT applications competing for their limited resources;
	\item \textbf{Virtualization support}. Introduces a software abstraction between the hardware and the OS and application running on the hardware;
	\item \textbf{Interoperability and federation}. Uses cooperation of different providers to support heavy applications such as real-time streaming. Moreover, it supports migration of applications to more suited fog servers depending on the current context;
	\item \textbf{Predominance of wireless access}. Most of the end devices only support wireless communication.
\end{itemize}

Nonetheless, as stated in Section \ref{subsec:Objectives}, fog still has some limitations. In order to tackle those limitations, it is necessary to understand its overall architecture. This includes understanding what are the actors and how they interact, how IoT nodes connect to the fog servers, how clients outsource the allocation and management of resources that they rely upon to these servers, how migration is performed, etc.\\
\noindent\tab Fig. \ref{fog_architecture} shows the typical fog computing architecture. As stated before, mist computing can be implemented in a layer between the fog servers and the end devices. Moreover, the presence of cloud servers is not imperative, however it is very important for numerous applications.

\begin{figure} [t]
	\centering
	\includegraphics[width=0.9\textwidth]{images/fog_architecture/fog_architecture}
	\caption{Typical architecture of fog computing.}
	\label{fog_architecture}
\end{figure}

\noindent\tab Fog computing layer is composed by fog nodes/servers, that allow the deployment of distributed, latency-aware applications and services. Those nodes can be either physical (e.g., gateways, switches, routers, servers) or virtual (e.g., virtualized switches, virtual machines, cloudlets) components that provide computing resources to end devices. They can be organized in clusters either vertically (to support isolation), horizontally (to support federation), or relative to fog nodes’ latency-distance to the IoT devices \cite{iorga2018fog}. Fog nodes can be accessed through connected devices located at the edge, which provide local computing resources and, when needed, provide network connectivity to centralized services (i.e. cloud). Moreover, fog nodes can operate in a centralized or decentralized manner and can be configured as stand-alone nodes. They are the fundamental components in this three tier architecture (i.e. IoT-fog-cloud), being able to support the six features shown below \cite{iorga2018fog}:
\begin{itemize}
	\item \textbf{Autonomy}. Fog nodes can be autonomous enough to operate independently, making local decisions, at the node or cluster-of-nodes level;
	\item \textbf{Heterogeneity}. Can be deployed in a wide variety of environments;
	\item \textbf{Hierarchical clustering}. The fog network can be organized with different numbers of layers, so that they are able to provide different subsets of service functions while working together as a continuum;
	\item \textbf{Manageability}. They are managed and orchestrated by complex systems that can perform most routine operations automatically;
	\item \textbf{Programmability}. Fog nodes are inherently programmable at multiple levels, by multiple stakeholders such as network operators, domain experts, equipment providers, or end users.
\end{itemize}

\noindent\tab Fog nodes generally are of most value in scenarios where data needs to be collected at the edge and where the data from thousands or even millions of devices is analyzed and acted upon in micro and milliseconds \cite{openfog2017openfog}. In order to being able to support such a large number of requests, especially those engaged in enhanced analytics, fog nodes may implement additional hardware. Accelerators modules (refer to Fig. \ref{fog_architecture}) can be implemented to provide supplementary computational throughput. For instance, hardware accelerators can be performed through Graphics Processing Units (GPUs); they are an optimal choice for applications that support parallelism or for stream processing. Also, fog nodes can opt to make use of Field Programmable Gate Arrays (FPGAs) or even Digital Signal Processors (DSPs) for this propose.\\
\noindent\tab It is worth noting that, once fog nodes can be anything with computational and storage power in the cloud-to-things continuum, the links formed in these architectures (i.e. End device-to-Fog, Fog-to-Fog and Fog-to-Cloud) can be of any type. For instance, end devices can be connected to fog servers by wireless access technologies (e.g., WLAN, WiFi, 3G, 4G, ZigBee, Bluetooth) or wired connection. Moreover, fog nodes can be interconnected by wired or wireless communication technologies and they can be linked into the cloud by IP core network.\\
\noindent\tab In this architecture, the connected sensors located at the edge, generate data that can adopt two models. First, in a sense-process-actuate model, the information collected is transmitted as data streams, which is acted upon by applications running on fog devices and the resultant commands are sent to actuators. In this model, the raw data collected often does not need to be transferred to the cloud; data can be processed, filtered, or aggregated in fog nodes, producing reduced data sets. The result can then be stored inside fog nodes or actuated upon through the actuators. Second, in a stream-processing model, sensors send equally data streams, where the information mined (from the incoming streams) is stored in data centers for large-scale and long-term analytics. In this case, big data needs to be stored and does not have that much latency constraints. Being fog servers less powerful than the cloud ones, cloud is far more suited for this kind of operations. Yet, fog servers can still shrink data, doing some intermediate processing as in the previous model. This meets the aforementioned statement - although cloud is not essential for the functioning of fog, in some applications it is beneficial or even essential.\\
\noindent\tab The applications deployed by the end users in fog nodes can be seen either as a whole or as a Distributed Data Flow (DDF) programming model, in which the applications are moduled as a collection of modules. DDF is proposed by N. Giang et al. \cite{giang2015developing} for IoT applications that utilizes computing infrastructures across the fog and the cloud, allowing the application flow to be deployed on multiple physical devices rather than one. This can be particular useful so that the less restricted modules in terms of latency can be deployed to the upper fog layers (ideally to the cloud), leaving the fog nodes of the lower layers less overloaded, being able to respond faster to modules within tighter latency bounds. As already mentioned, fog utilizes virtualization mechanisms due to the numerous advantages offered. Hence, hosting an application involves creating a set of VMs or execution containers (e.g., Docker) and assign them to a set of physical or virtual components along the cloud-to-things continuum.\\
\noindent\tab When an end device needs to offload some work to a third party, it needs somehow to know where to outsource the allocation and management of resources. For this propose, the fog architecture also needs a discovery service which concerns in finding the best available fog server, given certain capabilities and requirements. In this context, J. Gedeon et al. \cite{gedeon2017router}, propose a brokering mechanism in which available surrogates (i.e. fog nodes) advertise themselves to the broker. When it receives client requests, considering a number of attributes such as network information, hardware capabilities, and distance, it finds the best available surrogate for the client. This mechanism can be implemented on standard home routers, and thus, leverages the ubiquity of such devices in urban environments. Multiple brokers are interconnected using Distributed Hash Tables (DHTs) in order to exchange information.\\
\noindent\tab Finally, fog also needs an orchestration layer in order to monitor the current context and thus be able to make management decisions regarding applications and data placement. For this propose, F. Bonomi et al. \cite{bonomi2014fog} defines the fog orchestration layer which comprises the following technology and components: (1) a software agent, Foglet, with reasonably small footprint yet capable of bearing the orchestration functionality and performance requirements that could be embedded in various edge devices (2) a distributed, persistent storage to store policies and resource meta-data (capability, performance, etc) that support high transaction rate update and retrieval, (3) a scalable messaging bus to carry control messages for service orchestration and resource management, and (4) a distributed policy engine with a single global view and local enforcement.\\
%For this propose, E. Saurez et al. \cite{saurez2016incremental} propose foglets, a programming model that facilitates distributed programming across fog nodes. It is implemented through container-based visualization. Foglets not only provides APIs for application development as DDFs, including the primitives for communication between the application components, but also embodies algorithms for the discovery and incremental deployment of resources commensurate with the application needs. Moreover it provides mechanisms for QoS-sensitive and workload sensitive migration of application components due to end devices mobility and application dynamism. Specifically, there are four entities in the foglets runtime system: the discovery server, the docker registry server, the entry point daemon, and the worker process. The discovery server is a partitioned name server that maintains a list of fog nodes available. Docker registry server is a server that contains the binaries for the applications that have been launched on the foglets infrastructure. The entry point daemon executes directly on top of the host OS in the fog node, awaits requests and periodically sends ``I am alive'' message to the discovery server. Finally the worker process carries out the functionality contained in a particular application component assigned to it.\\
\noindent\tab Fog servers can provide reduced latencies and help in avoiding/reducing traffic congestion in the network core. However, this comes at a price: more complex and sophisticated resource management mechanisms are needed. This raises new challenges to be overcome such as dynamically deciding when, and where (device/fog/cloud) to carry out processing of requests to meet their QoS requirements. Furthermore, in mobile environments such mechanisms must incorporate mobility (i.e. location) of data sources, sinks and fog servers in the resource management and allocation process policies to promote and take advantage of proximity between fog and users.
%!TEX encoding = UTF-8 Unicode
\subsection{Migration Optimization in Mobile Fog Environments}
\label{sec:Migration}
\noindent When an IoT device needs to offload some heavy application to a third party, ideally it will be connected to the nearest server, securing an one-hop away fog server to ensure the shortest network delay. However, as their physical distance increases either by device or server movement, their network distance (i.e. the number of hops) will also increase. Hence, both latency and bandwidth usage by the intermediate links will increase, resulting in poor connectivity. This away, in such dynamic environments the decision-making of where to offload the work is a major concern. Moreover, even if both clients and servers are static, the end-to-end latency may increase due to unexpected crowds of mobile clients seeking to connect or making requests to the same fog server simultaneously, which may lead to QoS violations.\\
\noindent\tab In Section \ref{sec:fog_architecture} was verified that applications can be offloaded as a whole, or as a set of modules which may have different latency constraints. Regardless of their type, whenever it is justified the system needs to be readjusted. This is performed through the exchange of VMs or containers (containing the applications or modules) between fog nodes. For this reason, it is necessary to answer the following questions: \textit{When is this exchange justified? And what is the best placement for those applications and/or modules?} As stated in Section \ref{subsec:Objectives}, this work intends to implement multi-objective management decision-making in a novel architecture. Hence this state-of-the-art section intends to study some of the proposed mechanisms in the literature.

\subsubsection{Network Utilization-Aware}\label{NetworkUtilizationAware}
\noindent Minimization of network utilization is one of the main objectives of fog computing. In fact, fog appears to overcome this inherent limitation of cloud computing. Thus, besides guarantee QoS, namely the time boundaries of applications, it is also  important to reduce bandwidth usage. This utilization of network is essentially due to tree factors, the transmission of virtualized resources (VMs or containers), transmission of data between the client and the deployed application into the fog nodes, and control messages. If applications are deployed using DDF programming model, a fourth factor rises, the data transmission between modules. In this section, the reviewed literature propose models to mitigate the former, providing long term QoS, reducing the number of migrations.\\
\noindent\tab B. Ottenwälder et al. \cite{ottenwalder2013migcep} consider an environment with mobile devices and fixed fog nodes, where users offload real-time applications such as Complex Event Processing (CEP). CEP is a paradigm where changes in sensor measurements are modeled as events, while the application is modeled as set of event-driven operators. They state that each migration comes with a cost, consequence of the local state that also needs to be migrated along with the operators. Thus, frequent migration would significantly decrease the system performance. To overcome this limitation, they propose a placement and migration method for fog providers to support operator migrations in Mobile Complex Event Processing (MCEP) systems. Their method plans the migration ahead of time through knowledge of the MCEP system and predicted mobility patterns towards ensuring application-defined end-to-end latency restrictions and reducing the network utilization. These predicted mobility patterns were captured using three different methods: uncertain locations from the \textit{dead reckoning} approach (linear), certain locations that could stem from a \textit{navigation} system (navi), and \textit{learned} transitions between leaf broker (learned). This method, allows a minimization of migration costs by selecting migration targets that ensure a low expected network utilization for a sufficiently long time. Moreover, they present how the application knowledge of the CEP system can be used to improve current live migration techniques for VMs to reduce the required bandwidth during the migration (i.e. unnecessary events are not migrated). \\
\noindent\tab R. Urgaonkar et al. \cite{urgaonkar2015dynamic} argue that because of the uncertainty in user mobility and request patterns, it is challenging to make the decision in an optimal manner. Also, in this work is argued that methods which depend on mobility patterns have several drawbacks, namely: (1) it requires extensive knowledge of the statistics of the user mobility and request arrival processes that can be impractical to obtain in a dynamic network, (2) even when this is known, the resulting problem can be computationally challenging to solve, and (3) any change in the statistics would make the previous solution suboptimal and would require recomputing the optimal solution. Thus, they propose a new model, inspired by the technique of Lyapunov optimization, that overcomes these drawbacks (i.e. does not require any knowledge of the transition probabilities). The overall problem of dynamic service migration and workload scheduling to optimize system cost while providing end user performance guarantees is formulated as a sequential decision-making problem in the framework of Markov Decision Processes (MDPs). The cost depends on reconfiguration cost, inherent to migration services (i.e. moving the application from one cloudlet to another) and on transmission cost, time-average of user-to-cloudlet request routing that depends on their distance. They have developed a new approach for solving a class of constrained MDPs that possess a decoupling property. When this property holds, their approach enables the design of simple online control algorithms that do not require any knowledge of the underlying statistics of the MDPs.\\
\noindent\tab Also in this context, W. Zhang et al. \cite{zhang2016segue} state that previous studies have proposed a static distance-based MDP for optimizing migration decisions. However, these models fail to consider dynamic network and server states in migration decisions, assuming that all the important variables are known. Moreover, they also point out another unaddressed problem which lies in the recalculation time interval of the method. Since running MPD is a heavy computing task, a short recalculation interval introduces a considerable overhead to the server. On the other hand, a long recalculation interval may translate into lazy migration, resulting in periods of transgression of QoS guarantees. In order to overcome these issues, the authors propose SEGUE. This model achieves optimal migration decisions by providing a long-term optimal QoS to mobile users in the presence of link quality and server load variation. Additionally, SEGUE adopts a QoS aware scheme to activate the MDP model. In other words, it only activates the MDP model when QoS violation is predicted. Thus, it avoids unnecessary migration costs and bypass any possible QoS violations while keeping a reasonable low overhead in the servers. \\
\noindent\tab The work performed by Wuyang Zhang et al. \cite{zhang2017towards} use as case study the Massively Multiplayer Online Gamse (MMOGs) with Virtual Reality (VR) technologies, VR-MMOGs. They present the main challages of VR-MMOGs, namely: stringent latency, high bandwidth, and large scale requirements. This work shows one problem that remains unsolved: how to distribute the work among the user device, the fog nodes, and the center cloud to meet all three requirements especially when users are mobile. Their approach was to place local view change updates on fog nodes for immediate responses, frame rendering on fog nodes for high bandwidth, and global game state updates on the center cloud for user scalability. In this kind of games, users need to move, so in order to keep a low latency communication, they also propose an efficient service placement algorithm based on MDP. This method takes into account the presence of dynamic network states and server workload states, and user mobility providing long term QoS. To ensure feasibility of this method, they come up with an approach that reduces the algorithm complexity in both storage and execution time. Nonetheless, unlike many of the service migration solutions which assumes an ignorable service transition time, they point out that it is impossible to migrate an fog service from one fog to another instantly given the size of a VR game world. Therefore, they propose a mechanism to ensure a new fog node is activated when a player connects to the new one. \\[6pt]
\textit{Concluding Remarks} - The presented literature intends to reduce the number of migrations by providing long term QoS. This also reduces the downtime service, increasing QoS. While the first work, considers distributed applications, where the set of operators which are deployed among a set of fog nodes need some coordination in migrations, the rest consider applications as a whole. As already discussed, DDF programming model can bring advantages to fog computing. However the two latter works consider network state and server performance (resulting in workload) in their models. They all fail in not consider an energy consumption, a cost model and an environment where fog nodes can be movable.

\subsubsection{Energy-Aware}\label{EnergyAware}
\noindent In order to achieve the QoS objective, the placement of applications and their modules has often to be moved between different entities that compose the 3-tier architecture (things-fog-cloud) what evolves energetic costs. For instance, it is needed exchange control messages, communicate between modules placed at different nodes, change the module placement, etc. Thus, energy-aware must be an important factor to be taken into account in the decision making algorithm of \textit{when} and \textit{where} to offload work to another entity in order to increase fog infrastructure providers' profit.\\
\noindent\tab In this context, R. Deng et al. \cite{deng2016optimal} focused on investigating power consumption and network delay tradeoff in cloud-fog services. They formulate a workload allocation problem which suggests the optimal workload allocations between fog and cloud toward the minimal power consumption with the constrained service delay. This was performed trough the modeled power consumption function and delay function of each part of the fog-cloud computing system. It is worth noting that power consumption only considers energy consumption of work computation disregarding communication costs. The primal problem is then tackled using an approximate approach through decomposition, and formulate three subproblems of three corresponding subsystems. These subproblems can be solved via existing optimization techniques. \\
\noindent\tab Y. Xiao et al. \cite{xiao2017qoe} investigate two performance metrics for fog computing networks: mobile users’ QoS and fog nodes’ power efficiency. In their scheme fog nodes can process or offload to other fog nodes part of the workload that was initially sent to the cloud. Fog nodes decide to either offload the workload to neighbors or locally process it, under a given power constraint. A distributed optimization algorithm based on distributed Alternating Direction Method of Multipliers (ADMM) via variable splitting algorithm is proposed to achieve the optimal workload allocation solution that maximizes users’ QoS under the given power efficiency. In this work, power efficiency of each fog node is measured by the amount of power consumed by each fog node to offload each unit of workload from the cloud.\\
\noindent\tab C. Anglano et al. \cite{anglano2018profit} present the Online Profit Maximization (OPM) algorithm. It is an approximation algorithm that aims increasing the profit of fog infrastructure providers, by reducing the overall energy consumption of the infrastructure without a priori knowledge, and yet guarantee the QoS to its users. Their study considers mobile environments, where the end devices that are generating data are mobile.\\
\noindent\tab The work performed by A. Kattepur et al. \cite{kattepur2016resource} investigates the problem of computation offloading in fog computing. They present an energy model and communication costs with respect to computational offloading and formulate an optimal deployment strategy when dealing with distributed computation, while keeping energy and latency constraints in mind. The formulations are solved in Scilab using the Karmarkar linear optimization solver. They evaluate their approach in a sense-process-actuate model using a network of mobile robotic sensor-actuators developed in ROS/Gazebo.\\
\noindent\tab Y. Nan et al. \cite{nan2017adaptive} describe an online adaptive algorithm, Lyapunov Optimization on Time and Energy Cost (LOTEC), based on the technique of Lyapunov optimization. Their aim is to provide an energy-efficient data offloading mechanism to ensure minimization of long-term system cost (measured by the money spending on energy consumption) and yet guarantee that users do not experience a poor quality of service. This kind of problems, generally, can be converted into a constrained stochastic optimization problem. LOTEC is a quantified near optimal solution and is able to make control decision on application offloading by adjusting the two-way tradeoff. This decision-making distributes the incoming applications to the corresponding tiers without a priori knowledge of the status of users and system. In addition, their proposed algorithm achieves the average response time arbitrarily close to the theoretical
optimum. \\[6pt]
\textit{Concluding Remarks} - The above works aim to minimize the energy consumption of fog infrastructure providers. To this end they present efficient mechanisms regarding to offloading work in an energy-efficient manner while guarantee QoS to its users. Once again their models do not consider mobile fog nodes nor DDF programming model. Besides, these do not formulate a specific cost model or only consider energy consumption being its only variable.

\subsubsection{Cost-Aware}
\noindent As aforementioned besides guarantee QoS to its users, fog service providers also need to maximize their profit. Hence is important to develop an accurate cost model in order to accept and implement fog computing. Besides, similarly to what cloud does, fog has to implement a pay-as-you-go cost model in order to provide services on-demand to its users, without under- or over-provisioning, and charging a fair price. To this end, the cost model needs to apply a communication model, an energy model, and a resource utilization model.\\
\noindent\tab In this context, L. Gu et al. \cite{gu2017cost} state the importance of fog computing in medical cyber-physical systems as the number of users grows. In order to tackle the cost-efficiency problem in this kind of systems, they argue that communication Base Station (BS) association, task distribution and VM deployment are all critical. Therefore, in this paper, the authors jointly study these three issues towards minimizing the overall cost while satisfying the QoS requirement. They formulate the cost minimization problem in a form of Mixed-Integer NonLinear Programming (MINLP) with joint consideration of communication BS association, subcarrier allocation, computation BS association, VM deployment and task distribution. To tackle the high computational complexity of solving this problem, they linearize it into a Mixed-Integer Linear Programming (MILP) problem. This way they are able to solve the optimal programming model using solvers such as CPLEX and Gurobi. However, it is still time consuming due to the existence of a large number of integer variables. To this end, they further propose an LP-based two-phase heuristic algorithm. It is worth noting that this work explores placement of VMs in fog computing, whereas it does not tackle this problem in mobile environments, disregarding both users and servers mobility, consequently not addressing the inherit migration problems.\\
\noindent\tab Unlike the previous one, this work considers mobility of users. However, similarly the aim of L. Yang et al. \cite{yang2016cost} is to guarantee QoS to mobile users, minimizing the average latency of all the users’ request loads, while minimizing the overall costs of service providers. The latter, is composed by minimization of both resource usage on cloudlets and service placement transitions. The authors state that this three way trade-off is a difficult problem. Moreover, the request load could vary significantly and frequently in both spatial and temporal domain due to the mobility of users. Such dynamic request load implies a periodic update of decisions, keeping in mind both the current performance and affordable cost, and the expected future workload. In order to solve this tree way trade-off, the authors first formulate the snapshot problem, named Basic Service Placement Problem (BSPP), which aims to optimize the access latency with the capacity constraints of cloudlets. As it is hard to solve, they design a competitive heuristic to BSPP which outperforms a set of benchmark algorithms significantly in terms of the access latency, and algorithm run time. Further, they also extend BSPP to a more practical model that aims to optimize the trade-off between access latency, resource usage and service placement transitions. Finally they develop an online algorithm for this model that can be deployed directly in practical systems. It is worth noting that their solution utilizes user’s mobility pattern and services access pattern to predict the distribution of user’s future requests, and then adapt the service placement and load dispatching online based on the prediction. Their work does not consider DDF programming model, an energy model nor mobility of fog nodes.\\
\noindent\tab O. Skarlat et al. \cite{skarlat2017optimized} start by describing a conceptual framework for resource provisioning and service placement in fog. They consider the concept of fog colonies, micro-data centers made up from an arbitrary number of fog cells. Within a fog colony, services and data can be distributed and shared between the single cells, leading to a cooperative execution of IoT applications (DDF programming model). Based on this concept, their work, formalizes an optimization problem that aims to adhere to the deadlines on deployment and execution time of applications and to maximize the utilization of existing resources in fog, rather than in cloud, leading to lower execution cost. To solve this placement problem, they apply different approaches, namely the exact optimization method and its approximation through a greedy first fit heuristic and a genetic algorithm. They also compare the results, in the fog simulation toolkit iFogSim, to a classical approach that neglects fog resources and runs all services in a centralized cloud. The goal of the evaluation is to identify the best approach to solve the proposed optimization problem in terms of resulting QoS, QoS violations, and cost. This work does not provide mobility mechanisms, not addressing the inherit migration problems nor communication energetic/monetary costs. \\
\noindent\tab L. Wang et al. \cite{wang2018service} address the social VR applications to study the problem of placing service entities (VMs deployed) in fog environments. Although motivated by VR applications, the authors state that this problem is fundamental for any applications that require interactions between either user and the respective VM or user and VMs of other users. The placement problem is to decide where to place the service entity of each user among the cloudlets in order to achieve economic operations of the cloudlets as well as satisfactory QoS for the users. This problem is is non-trivial due to the following challenges: (1) cloudlets are heterogeneous, (2) VMs need to exchange metadata frequently with the associated users and other VMs (of other users), due to the type of application in cause, and (3) due to the fact that cloudlets are not intentionally designed to simultaneously accommodate a large number of VMs, especially for VR applications where specific hardware such as GPU may be involved, resource contention needs to be controlled. The authors model the aforementioned challenges with four types of cost: activation cost, the placement cost, the proximity cost, and the collocation cost. The considered cost models are comprehensive yet practical, and are general enough to capture a wide range of concrete performance measures in reality. They formulate the problem as a combinatorial optimization, which is NP-hard. To solve the problem, they propose ITerative Expansion Moves (ITEM) algorithm, a novel algorithm based on iteratively solving a series of minimum graph cuts. The algorithm is flexible and is applicable in both offline and online cases. It is worth noting that in this case there is only one VM deployed to the fog per application (i.e. the concept of DDF is not applied in this work). Also it does not consider any specific energy model.\\
\noindent\tab The work performed by T. Bahreini et al. \cite{bahreini2017efficient} also addresses multi tier placement (DDF programming model). The authors formulate the Multi-Component Application Placement Problem (MCAPP). Their objective is to find a mapping between components and servers, such that the total placement cost is minimized. This cost is composed of four types of costs at each time slot: (1) the cost of running one component in a specific server, (2) cost of relocating one component from one server to another, (3) communication cost between one component and the user, and (4) communication between components. With the objective to minimize the overall cost incurred when running the application, they we formulated the offine version of the problem as a Mixed Integer Linear Program (MILP) and then developed a heuristic algorithm for solving the online version of the problem. The algorithm is based on an iterative matching process followed by a locals search phase in which the solution quality is improved. This way they use simple algorithmic techniques, avoiding complex approaches such as those based on MDPs. They state that the proposed algorithm has low complexity and adds a negligible overhead to the execution of the applications. Although this work considers the location of servers in the estimation of cost (2), it does not consider an environment with mobile fog nodes.\\
\noindent\tab The study performed by X. Sun et al. \cite{sun2016primal} presents, similarly to the previous ones, a case scenario where end devices are mobile. To preform this work they use a cloudlet network architecture to bring the computing resources from the centralized cloud to the edge. They present the PRofIt Maximization Avatar pLacement (PRIMAL) strategy. PRIMAL maximizes the trade-off between the migration gain (i.e. the end-to-end delay reduction) and the migration cost (i.e. the migration overheads), selectively migrating the avatars (a software clone located in a cloudlet) to their optimal locations.\\
\noindent\tab A different approach was taken by D. Ye et al. \cite{ye2016scalable}. They leverage the characteristics of buses and propose a scalable fog computing paradigm with servicing offloading in bus networks. Knowing that buses have fixed mobility trajectories and strong periodicity, they consider a fog computing paradigm with service offloading in bus networks which is composed by two parts: roadside cloudlets and bus fog servers. The roadside cloudlet consists of three components: dedicated local servers, location-based service (LBS) providers, access points (APs). The dedicated local servers virtualize physical resource and act as a potential cloud computing site. The LBS providers offer the real time location of each bus in bus networks. The APs act as gateways for mobile users and bus fog servers within the communication coverage to access the roadside cloudlet. When users need to offload some computationally intensive and delay sensitive tasks, they access APs and use the computing service of the roadside cloudlet. However, as cloudlets have limited computational and storage resources, they may became overloaded. The bus fog server is a virtualized computing system on bus, which is similar to a light-weight cloudlet server. Hence, those buses not only provide fog computing services for the mobile users on bus, but also are motivated to accomplish the computation tasks offloaded by roadside cloudlets. This allocation strategy is accomplished using genetic algorithm (GA), where the objective is to minimize the cost that roadside cloudlets spend to offload their computation tasks. Meanwhile, the user experience of mobile users are maintained. Although this work refers to mobile users, its meaning is not literal (representing the workload of both the buses and the roadside cloudlets), being supported only the mobility of fog servers. \\[6pt] %In their problem formulation there are two types of mobile users. On one hand there are mobile users that already have offloaded their computing tasks to the roadside cloudlets (i.e. representing the workload of the cloudlets). On the other hand there are several mobile users inside the bus that have also offloadded their tasks (i.e. representing the workload of bus fog servers).
\textit{Concluding Remarks} - Cost-aware is the most comprehensive and addressed in the literature. Unlike Section \ref{NetworkUtilizationAware} and Section \ref{EnergyAware} that treat singular problems, these works are much closer that what this work wants to implement. However, these works do not consider all necessary variables to implement a good and realistic fog placement model. For instance, none of them address the mobility of both end devices and fog nodes. Some of them do not consider DDF programming model. Also, most of the works penalize migration, and considers that since QoS is being guaranteed, there is no need to migrate applications/modules, but as network distance increases, the bandwidth usage by the intermediate links also increases, thus this trade-off needs to be carefully implemented. 

\subsubsection{Handover}
\noindent\tab X. Sun et al. \cite{sun2017avaptive} shows an architecture where each User Equipment (UE) has its own Avatar (a private computing and storage resources for the UE) which is deployed to a cloudlet, being the communication characterized by low end-to-end (E2E) latency. When UEs roam away, in order to maintain the end-to-end latency, their Avatars should be handed off among cloudlets accordingly. However, moving such amount of data (the Avatar’s virtual disk) during the handoff time may both incur unbearable migration time and network congestion. In order to overcome those limitations, they propose LatEncy Aware Replica placemeNt (LEARN) algorithm to place a number of replicas of each Avatar’s virtual disk into suitable cloudlets. Meanwhile, by considering the capacity limitation of each cloudlet, they propose the LatEncy aware Avatar hanDoff (LEAD) algorithm to place
UEs’ Avatars among the cloudlets such that the average E2E delay is minimized.\\ %nao sei se devia estar aqui!
\cite{bao2017follow}
The authors observe that traditional mobile network handover mechanisms cannot handle the demands of fog computation resources and the low-latency requirements of mobile IoT applications. The authors propose Follow Me Fog framework to guarantee service continuity and reduce latency during handovers. The key idea proposed is to continuously monitor the received signal strength of the fog nodes at the mobile IoT device, and to trigger pre-migration of computation jobs before disconnecting the IoT device from the existing fog node.\\
\cite{ma2017efficient}
Present a novel service handoff system which seamlessly migrates offloading services to the nearest edge server, while the mobile client is moving. Service handoff is achieved via container migration. They have identified an important performance problem during Docker container migration, proposing a migration method which leverages the layered storage system to reduce file system synchronization overhead, without dependence on the distributed file system.\\
\cite{sun2017emm}
Develop a novel user-centric energy-aware mobility management (EMM) scheme, in order to optimize the delay, under energy consumption constraint of the user. Based on Lyapunov optimization and multi-armed bandit theories, EMM works in an online fashion. Theoretical analysis explicitly takes radio handover and computation migration cost into consideration and proves a bounded deviation on both the delay performance and energy consumption compared with the oracle solution with exact and complete future system information. The proposed algorithm also effectively handles the scenario in which candidate BSs randomly switch ON/OFF during the offloading process of a task.\\
\cite{bi2018mobility}
Study of mobility support issue in fog computing for guaranteeing service continuity. Propose a novel SDN enabled architecture that is able to facilitate mobility management in fog computing by decoupling mobility management and data forwarding functions. Design an efficient handover scheme by migrating mobility management and route optimization logic to the SDN controller. By employing link layer information, the SDN controller can pre-compute the optimal path by estimating the performance gain of each path.\\
\cite{farris2017optimizing}
To guarantee the strict latency requirements, new solutions are required to cope with the user mobility in a distributed edge cloud environment. The use of proactive replication mechanism seems promising to avoid QoE degradation during service migration between different edge nodes. However, accounting for the limited resources of edge micro data-centers, appropriate optimization solutions must be developed to reduce the cost of service deployment, while guaranteeing the desired QoE. In this paper, Ivan Farris et al., by leveraging on prediction schemes of user mobility patterns, have proposed two linear optimization solutions for replication-based service migration in cellular 5G networks: the min-RM approach aims at minimizing the QoE degradation during user handover; min-NSR approach favors the reduction of service replication cost. Simulation results proved the efficiency of each solution in achieving its design goal and provides useful information for network and service orchestrators in next-generation 5G cloud-based networks.
%\noindent\tab Fog computing will be crucial in a diversity of scenarios. For
%instance, heterogeneous sensory nodes (e.g., sensors, controllers, actuators)
%on a self-driving vehicle, are estimated to generate about 1GB data per second
%\cite{angelica2013google}. As the number of features grow, the data deluge
%grows out of control. Moreover, these types of systems, where people's lives
%depends on it, are hard real-time what means that it is absolutely imperative
%that all deadlines are met. Offloading tasks to fog nodes will be the best
%solution, once a big effort in mobility support has been done through the
%migration of VMs using cloudlets \cite{lopes2017myifogsim}. Also, in this
%context, Puliafito et al. address three types of applications where fog is
%required, namely, citizen's healthcare, drones for smart urban surveillance and
%tourists as time travellers \cite{puliafito2017fog}, addressing the needs of
%low latency and mobility support.\\

%%Previous reactive load balancing algorithms migrate VMs upon the occurrence of load imbalance, while previous proactive load balancing algorithms predict PM overload to conduct VM migration
% another works were performd using MDP-based approaches (ex. Distributed Autonomous Virtual Resource Management in Datacenters Using Finite-Markov Decision Process)
%!TEX encoding = UTF-8 Unicode
\section{Toolkits}
\label{sec:Toolkits}
As stated in Section \ref{subsec:Objectives}, the proposed solution, which will be described later in Section \ref{sec:Architecture}, will be implemented in a carefully selected toolkit. In order to perform this selection, a survey was made on the currently available simulators. Table \ref{tab:toolkits} compares fog and related computing paradigm simulators via comparison of their characteristics.

\begin{table}[h]
	\caption{Comparison of fog and related computing paradigms simulators (`*' - extends CloudSim, `**' - extends iFogSim, `\halfcorrect' - limited).}
	\scriptsize
	\begin{tabular}{>{\arraybackslash}m{1in} >{\centering\arraybackslash}m{0.2in} >{\centering\arraybackslash}m{0.19in} >{\centering\arraybackslash}m{0.19in} >{\centering\arraybackslash}m{0.19in} >{\centering\arraybackslash}m{0.19in} >{\centering\arraybackslash}m{0.19in} >{\centering\arraybackslash}m{0.19in} >{\centering\arraybackslash}m{0.19in} >{\centering\arraybackslash}m{0.19in} >{\centering\arraybackslash}m{0.19in} >{\centering\arraybackslash}m{0.19in} >{\centering\arraybackslash}m{0.33in} >{\centering\arraybackslash}m{0.19in} >{\centering\arraybackslash}m{0.19in}}
		\toprule
		Simulation Platform &
		\begin{turn}{90}\shortstack{Programming\\language}\end{turn} &
		\begin{turn}{90}\shortstack{Documentation}\end{turn} &
		\begin{turn}{90}\shortstack{Graphical support}\end{turn} &
		\begin{turn}{90}\shortstack{Energy-aware}\end{turn} &
		\begin{turn}{90}\shortstack{Cost-aware}\end{turn} &
		\begin{turn}{90}\shortstack{Virtual machine\\support}\end{turn} &
		\begin{turn}{90}\shortstack{Application\\models}\end{turn} &
		\begin{turn}{90}\shortstack{Communication\\model}\end{turn} &
		\begin{turn}{90}\shortstack{Migration support}\end{turn} &
		\begin{turn}{90}\shortstack{Mobility/\\Location-aware}\end{turn} &
		\begin{turn}{90}\shortstack{Fog/Edge support}\end{turn} &
		\begin{turn}{90}\shortstack{Last commit}\end{turn} &
		\begin{turn}{90}\shortstack{Web page}\end{turn} &
		\begin{turn}{90}\shortstack{Paper}\end{turn} \\
		\toprule
		CloudSim & Java & \cmark &  & \cmark & \cmark & \cmark & \cmark & \halfcorrect & \cmark &  &  & 2016 &  \cite{TheCLOUD47:online} & \cite{calheiros2011cloudsim} \\ \midrule
		CloudNetSim++ & C++ &  & \cmark & \cmark & \cmark & \cmark & \cmark & \cmark & \cmark &  &  & 2015 & \cite{cloudnet14:online} & \cite{malik2017cloudnetsim++} \\ \midrule
		GreenCloud & \shortstack{C++/\\ Otcl} & \cmark &  & \cmark &  & \cmark & \cmark & \cmark & \cmark &  &  & 2016 & \cite{Greenclo13:online} & \cite{kliazovich2012greencloud} \\ \midrule
		iCanCloud & C++ & \cmark & \cmark & \cmark & \cmark & \cmark & \cmark & \cmark &  &  &  & 2015 & \cite{Website18:online} & \cite{nunez2012icancloud} \\ \midrule
		CloudSched & Java &  & \cmark & \cmark &  & \cmark &  &  &  &  &  & 2015 & \cite{CloudSch23:online} & \cite{tian2015toolkit} \\ \midrule
		CloudAnalyst* & Java &  & \cmark &  & \cmark & \cmark & \cmark & \halfcorrect & \cmark & \cmark &  & 2009 & \cite{TheCLOUD47:online} & \cite{wickremasinghe2010cloudanalyst} \\ \midrule
		DynamicCloudSim* & Java &  &  & \cmark & \cmark & \cmark & \cmark & \halfcorrect & \cmark &  &  & 2017 & \cite{marcbuxd5:online} & \cite{bux2015dynamiccloudsim} \\ \midrule
		CloudReports* & Java &  & \cmark & \cmark & \cmark & \cmark & \cmark & \halfcorrect & \cmark &  &  & 2012 & \cite{thiagott93:online} & \cite{sa2014cloudreports} \\ \midrule
		RealCloudSim* & Java & \halfcorrect & \cmark & \cmark & \cmark & \cmark & \cmark & \cmark & \cmark &  &  & 2013 & \cite{RealClou60:online} & \\ \midrule
		DCSim & Java & \halfcorrect &  & \cmark &  & \cmark & \cmark & \halfcorrect & \cmark &  &  & 2014 & \cite{digsuwod49:online} & \cite{tighe2012dcsim} \\ \midrule
		CloudSim Plus* & Java & \cmark &  & \cmark & \cmark & \cmark & \cmark & \cmark & \cmark &  &  & 2018 & \cite{CloudSim79:online} & \cite{silva2017cloudsim} \\ \midrule
		CloudSim Plus Automation* & Java & \cmark &  & \cmark & \cmark & \cmark & \cmark & \cmark & \cmark &  &  & 2018 & \cite{manoelca57:online} & \\ \midrule
		DISSECT-CF & Java & \cmark &  & \cmark &  & \cmark &  & \halfcorrect & \cmark &  &  & 2018 & \cite{kecskeme90:online} & \cite{kecskemeti2015dissect} \\ \midrule
		WorkflowSim* & Java & \cmark &  & \cmark & \cmark & \cmark & \cmark & \halfcorrect & \cmark &  &  & 2015 & \cite{Workflow31:online} & \cite{chen2012workflowsim} \\ \midrule
		Cloud2Sim* & Java &  &  & \cmark & \cmark & \cmark & \cmark & \halfcorrect & \cmark &  &  & 2016 & \cite{Cloud2Si98:online} & \cite{kathiravelu2014adaptive} \\ \midrule
		CloudSimDisk* & Java &  &  & \cmark & \cmark & \cmark & \cmark & \halfcorrect & \cmark &  &  & 2015 & \cite{Udacity231:online} & \cite{louis2015cloudsimdisk} \\ \midrule
		iFogSim* & Java & \cmark & \cmark & \cmark & \cmark & \cmark & \cmark & \halfcorrect &  &  & \cmark & 2016 & \cite{Cloudsla14:online} & \cite{gupta2017ifogsim} \\ \midrule
		MyiFogSim** & Java &  & \cmark & \cmark & \cmark & \cmark & \cmark & \halfcorrect & \cmark & \cmark & \cmark & 2017 & \cite{marcioco38:online} & \cite{lopes2017myifogsim} \\ \midrule
		iFogSimWithData Placement** & Java &  & \cmark & \cmark & \cmark & \cmark & \cmark & \halfcorrect &  &  & \cmark & 2018 & \cite{medislam49:online} & \cite{naas2018extension} \\ \midrule
		EdgeCloudSim & Java & \cmark &  &  & \cmark & \cmark & \cmark & \cmark &  & \cmark & \cmark & 2018 & \cite{CagatayS20:online} & \cite{sonmez2017edgecloudsim} \\ \midrule
		YAFS & \shortstack{Pyth-\\on} & \cmark &  & \halfcorrect &  &  & \halfcorrect & \halfcorrect &  &  & \cmark & 2018 & \cite{yafsPyP38:online} & \\ \midrule
		FogTorch & Java &  &  &  &  & \cmark & \cmark & \halfcorrect &  &  & \cmark & 2016 & \cite{diunipis47:online} & \cite{brogi2017qos} \\ \bottomrule
	\end{tabular}
	\label{tab:toolkits}
\end{table}

\begin{itemize}[noitemsep]
\item \textbf{Programming language}. This is important to evaluate the simplicity, level of abstraction offered, maintainability, extensibility, its popularity, etc. As can be observed, almost all are Java-based, being that all opt for object-oriented programming;
\item \textbf{Documentation}. Unlike the availability, documentation is not always available or sometimes is scarce. In those cases, it is an impediment to the extensibility and maintenance of the corresponding simulators. This parameter includes official documentation, tutorials, community, wiki, etc;
\item \textbf{Graphical support}. Provide a Graphical User Interface (GUI) may be helpful. Instead of defining the entire architecture programmatically, researchers can define it in a user-friendly environment;
\item \textbf{Energy-aware}. As aforementioned, energy is one of the multi-objectives that this work intends to cover. When implementing the migration optimization algorithm, the more realistic the energy model, the more realistic the algorithm will be. Although CloudSim provides energy-conscious resource management techniques/policies (supports modeling and simulation of different power consumption models and power management techniques), GreenCloud is a more fine-grained simulator to this end. Its energy models are implemented for every data center element. Moreover, due to the advantage in the simulation resolution, energy models can operate at the packet level as well. This allows updating the levels of energy consumption whenever a new packet leaves or arrives from the link, or whenever a new task execution is started or completed at the server \cite{kliazovich2012greencloud};
\item \textbf{Cost-aware}. Similarly, cost-aware is also an important parameter. It is related to the execution of tasks, bandwidth usage and the energy spent during the migrations. Thus, a trade-off between QoS and cost has to be defined. Moreover, migration results in an increase usage of computing resources that are performing non-useful work (overhead). Therefore, a cost model referring to the quantification in monetary terms of the usage of infrastructure service providers' resources is important, once it allows to apply a pay-as-you go model;
\item \textbf{Application models}. This is an important feature in terms of QoS because it allows specifying the computational requirements for the application and a specific completion deadline;
\item \textbf{Communication model}. CloudSim can model network components, such as switches, but lacks fine-grained communication models of links and Network Interface Cards (NIC) causing VM migration and packet simulation to be network-unaware \cite{malik2017cloudnetsim++}. CloudNetSim++, on the other hand, supports a simulation model of real physical network characteristics such as network congestion, packet drops, bit error, and packet error rates. Moreover, GreenCloud allows communications based on TCP/IP protocol. It allows capturing the dynamics of widely used communication protocols such as IP, TCP, UDP, etc. Whenever a message needs to be transmitted between two simulated elements, it is fragmented into a number of packets bounded in size by network Maximum Transmission Unit (MTU). Then, while routed in the data center network, these packets become a subject to link errors or congestion-related losses in network switches \cite{kliazovich2012greencloud};
\item \textbf{Migration support}. This policy allows applying data placement techiniques (i.e. application and workload migration) to benefit high QoS;
\item \textbf{Mobility/Location-aware}. As already explained, mobility/location-aware is quite an essential feature in fog computing. It allows maintaining (as much as possible) the end-to-end latency as both users and servers move. There are few simulators that support this feature. For instance, in cloud environments, CloudAnalyst \cite{wickremasinghe2010cloudanalyst} is a tool whose goal is to support the evaluation of social network applications, according to the geographic distribution of users and data centers. However, in fog environments, to the best of our knowledge, there is no support for mobility of fog nodes. The only that provides mobility/location-aware that is currently available, is MyiFogSim. It is an extension of iFogSim to support users' mobility through migration of VMs between cloudlets \cite{lopes2017myifogsim}.
\end{itemize}