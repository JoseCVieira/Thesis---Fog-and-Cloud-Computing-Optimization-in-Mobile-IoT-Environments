%!TEX encoding = UTF-8 Unicode
\subsection{Data placement}
\label{sec:Data_placement}

\cite{saurez2016incremental}
Enrique Saurez et al. propose Foglets, a programming model that facilitates distributed programming across fog nodes. Foglets provides APIs for spatio-temporal data abstraction for storing and retrieving application-generated data on the local nodes. Through the API, Foglets processes are set for a certain geospatial region and Foglets manages the application components on the Fog nodes. Foglets is implemented through container-based visualization. The API takes into account QoS and load balancing when migrating persistent (stateful) data between fog nodes. It provides various functionalities: automatically discovers fog computing resources deploys application components onto the fog computing resources commensurate with the latency requirements of each component in the application. It supports multi-application collocation on any compute node. Provides communication APIs for components that are deployed at different physical levels of the network hierarchy to communicate with one another to exchange application state. Lastly, it supports both latency- and workload-driven resource adaptation and state migration over space (geographic) and time to deal with the dynamism in situation awareness application.\\

\cite{li2018virtual}
They propose a layered Fog framework to better support IoT applications through virtualization. The virtualization is divided into object virtualization  (VOs), network function virtualization and service virtualization. VOs to address the protocol inconsistency (lack of unified networking protocols that leads to exaggerated overhead); Network function virtualization maps standard networking services to VOs, thus, minimize the communication process between consumers and producers by minimizing latency, improving security and scalability; Service virtualization that composes the community and Cloud Apps from various vendors to serve local Fog users with high quality of experience (QoE) but at low cost. At last, Foglets are involved to seamless aggregate multiple independent virtual instances, Fog network infrastructures, and software platforms.\\

\cite{giang2015developing}
This paper proposes a Distributed Dataflow (DDF) programming model for the IoT that utilizes computing infrastructures across the Fog and the Cloud. Also, evaluate their proposal by implementing a DDF framework based on Node-RED (Distributed Node-RED or D-NR), a visual programming tool that uses a flow-based model for building IoT applications. To address challenges of the intrinsic nature of the IoT (heterogeneous devices/resources, a tightly coupled perception-action cycle and widely distributed devices and processing), they propose a Distributed Dataflow (DDF) programming model for the IoT that utilities computing infrastructures across the Fog and the Cloud. Also, they evaluate their proposal by implementing a DDF framework based on Node-RED (Distributed Node-RED or D-NR), a visual programming tool that uses a flow-based model for building IoT applications.\\

\cite{bahreini2017efficient}
The authors address the problem of multi-component application placement on fog nodes. Each application could be modeled as a graph, where each node is a component of the application, and the edges indicate the communication between them.\\


\cite{bruschi2018move}
With the state-of-the-art virtualization technologies, services can be implemented in modular software as a graph/chain of portable VOs that can be dynamically migrated around the Telco infrastructure. It is proposed a VO clustering and migration policy that jointly considers user proximity and inter-VO affinity to scalably support user mobility, while allowing service differentiation among users.\\
%\subsubsection{Virtual Objects} \label{subsec:VirtualObjects}
%\subsubsection{Virtual Machines} \label{subsec:VirtualMachines}