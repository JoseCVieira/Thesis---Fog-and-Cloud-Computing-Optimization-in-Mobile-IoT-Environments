%!TEX encoding = UTF-8 Unicode
\subsection{Toolkits}
\label{sec:Toolkits}
\noindent As stated in Section \ref{subsec:Objectives}, the proposed solution, which will be described later in Section \ref{sec:Architecture}, will be implemented in a carefully selected toolkit. In order to preform this selection, a survey was made on the currently available simulators. Table \ref{tab:toolkits} compares fog and related computing paradigms simulators via comparison of their characteristics.\\
\noindent\tab \textit{Programming language}: This is important to evaluate the simplicity, level of abstraction offered, maintainability, extensibility, its popularity, etc. As can be observed, almost all are Java-based, being that all opt for object-oriented programming.\\
\begin{table}[!t]
	\caption{Comparison of fog and related computing paradigms simulators (`*' - extends CloudSim, `**' - extends iFogSim, `\halfcorrect' - limited).}
	\scriptsize
	\begin{tabular}{>{\arraybackslash}m{1in} >{\centering\arraybackslash}m{0.33in} >{\centering\arraybackslash}m{0.27in} >{\centering\arraybackslash}m{0.27in} >{\centering\arraybackslash}m{0.27in} >{\centering\arraybackslash}m{0.27in} >{\centering\arraybackslash}m{0.27in} >{\centering\arraybackslash}m{0.27in} >{\centering\arraybackslash}m{0.27in} >{\centering\arraybackslash}m{0.27in} >{\centering\arraybackslash}m{0.27in} >{\centering\arraybackslash}m{0.27in} >{\centering\arraybackslash}m{0.33in} >{\centering\arraybackslash}m{0.27in} >{\centering\arraybackslash}m{0.27in}}
		\toprule
		Simulation Platform &
		\begin{turn}{90}\shortstack{Programming\\language}\end{turn} &
		\begin{turn}{90}\shortstack{Documentation}\end{turn} &
		\begin{turn}{90}\shortstack{Graphical support}\end{turn} &
		\begin{turn}{90}\shortstack{Energy-aware}\end{turn} &
		\begin{turn}{90}\shortstack{Cost-aware}\end{turn} &
		\begin{turn}{90}\shortstack{Virtual machine\\support}\end{turn} &
		\begin{turn}{90}\shortstack{Application\\models}\end{turn} &
		\begin{turn}{90}\shortstack{Communication\\model}\end{turn} &
		\begin{turn}{90}\shortstack{Migration support}\end{turn} &
		\begin{turn}{90}\shortstack{Mobility/\\Location-aware}\end{turn} &
		\begin{turn}{90}\shortstack{Fog/Edge support}\end{turn} &
		\begin{turn}{90}\shortstack{Last commit}\end{turn} &
		\begin{turn}{90}\shortstack{Web page}\end{turn} &
		\begin{turn}{90}\shortstack{Paper}\end{turn} \\
		\toprule
		CloudSim & Java & \cmark &  & \cmark & \cmark & \cmark & \cmark & \halfcorrect & \cmark &  &  & 2016 &  \cite{TheCLOUD47:online} & \cite{calheiros2011cloudsim} \\ \midrule
		CloudNetSim++ & C++ &  & \cmark & \cmark & \cmark & \cmark & \cmark & \cmark & \cmark &  &  & 2015 & \cite{cloudnet14:online} & \cite{malik2017cloudnetsim++} \\ \midrule
		GreenCloud & \shortstack{C++/\\ Otcl} & \cmark &  & \cmark &  & \cmark & \cmark & \cmark & \cmark &  &  & 2016 & \cite{Greenclo13:online} & \cite{kliazovich2012greencloud} \\ \midrule
		iCanCloud & C++ & \cmark & \cmark & \cmark & \cmark & \cmark & \cmark & \cmark &  &  &  & 2015 & \cite{Website18:online} & \cite{nunez2012icancloud} \\ \midrule
		CloudSched & Java &  & \cmark & \cmark &  & \cmark &  &  &  &  &  & 2015 & \cite{CloudSch23:online} & \cite{tian2015toolkit} \\ \midrule
		CloudAnalyst* & Java &  & \cmark &  & \cmark & \cmark & \cmark & \halfcorrect & \cmark & \cmark &  & 2009 & \cite{TheCLOUD47:online} & \cite{wickremasinghe2010cloudanalyst} \\ \midrule
		DynamicCloudSim* & Java &  &  & \cmark & \cmark & \cmark & \cmark & \halfcorrect & \cmark &  &  & 2017 & \cite{marcbuxd5:online} & \cite{bux2015dynamiccloudsim} \\ \midrule
		CloudReports* & Java &  & \cmark & \cmark & \cmark & \cmark & \cmark & \halfcorrect & \cmark &  &  & 2012 & \cite{thiagott93:online} & \cite{sa2014cloudreports} \\ \midrule
		RealCloudSim* & Java & \halfcorrect & \cmark & \cmark & \cmark & \cmark & \cmark & \cmark & \cmark &  &  & 2013 & \cite{RealClou60:online} & \\ \midrule
		DCSim & Java & \halfcorrect &  & \cmark &  & \cmark & \cmark & \halfcorrect & \cmark &  &  & 2014 & \cite{digsuwod49:online} & \cite{tighe2012dcsim} \\ \midrule
		CloudSim Plus* & Java & \cmark &  & \cmark & \cmark & \cmark & \cmark & \cmark & \cmark &  &  & 2018 & \cite{CloudSim79:online} & \cite{silva2017cloudsim} \\ \midrule
		CloudSim Plus Automation* & Java & \cmark &  & \cmark & \cmark & \cmark & \cmark & \cmark & \cmark &  &  & 2018 & \cite{manoelca57:online} & \\ \midrule
		DISSECT-CF & Java & \cmark &  & \cmark &  & \cmark &  & \halfcorrect & \cmark &  &  & 2018 & \cite{kecskeme90:online} & \cite{kecskemeti2015dissect} \\ \midrule
		WorkflowSim* & Java & \cmark &  & \cmark & \cmark & \cmark & \cmark & \halfcorrect & \cmark &  &  & 2015 & \cite{Workflow31:online} & \cite{chen2012workflowsim} \\ \midrule
		Cloud2Sim* & Java &  &  & \cmark & \cmark & \cmark & \cmark & \halfcorrect & \cmark &  &  & 2016 & \cite{Cloud2Si98:online} & \cite{kathiravelu2014adaptive} \\ \midrule
		CloudSimDisk* & Java &  &  & \cmark & \cmark & \cmark & \cmark & \halfcorrect & \cmark &  &  & 2015 & \cite{Udacity231:online} & \cite{louis2015cloudsimdisk} \\ \midrule
		iFogSim* & Java & \cmark & \cmark & \cmark & \cmark & \cmark & \cmark & \halfcorrect &  &  & \cmark & 2016 & \cite{Cloudsla14:online} & \cite{gupta2017ifogsim} \\ \midrule
		MyiFogSim** & Java &  & \cmark & \cmark & \cmark & \cmark & \cmark & \halfcorrect & \cmark & \cmark & \cmark & 2017 & \cite{marcioco38:online} & \cite{lopes2017myifogsim} \\ \midrule
		iFogSimWithData Placement** & Java &  & \cmark & \cmark & \cmark & \cmark & \cmark & \halfcorrect &  &  & \cmark & 2018 & \cite{medislam49:online} & \cite{naas2018extension} \\ \midrule
		EdgeCloudSim & Java & \cmark &  &  & \cmark & \cmark & \cmark & \cmark &  & \cmark & \cmark & 2018 & \cite{CagatayS20:online} & \cite{sonmez2017edgecloudsim} \\ \midrule
		YAFS & \shortstack{Pyth-\\on} & \cmark &  & \halfcorrect &  &  & \halfcorrect & \halfcorrect &  &  & \cmark & 2018 & \cite{yafsPyP38:online} & \\ \midrule
		FogTorch & Java &  &  &  &  & \cmark & \cmark & \halfcorrect &  &  & \cmark & 2016 & \cite{diunipis47:online} & \cite{brogi2017qos} \\ \bottomrule
	\end{tabular}
	\label{tab:toolkits}
\end{table}
\noindent\tab \textit{Documentation}: Unlike the availability, documentation is not always available or sometimes is scarce. In those cases, it is an impediment to the extensibility and maintenance of the corresponding simulators. This parameter includes official documentation, tutorials, community, wiki, etc.\\
\noindent\tab \textit{Graphical support}: Provide a Graphical User Interface (GUI) may be helpful. Instead of defining the entire architecture programmatically, researchers can define it in a user-friendly environment.\\
\noindent\tab \textit{Energy-aware}: As aforementioned, energy is one of the multi-objectives that this work intends to cover. When implementing the migration optimization algorithm, the more realistic the energy model, the more realistic the algorithm will be. Although CloudSim provides energy-conscious resource management techniques/policies (support modeling
and simulation of different power consumption models and power management techniques), GreenCloud is a more fine-grained simulator to this end. Its energy models are implemented for every data center element (computing servers, core and rack switches). Moreover, due to the advantage in the simulation resolution, energy models can operate at the packet level as well. This allows updating the levels of energy consumption whenever a new packet leaves or arrives from the link, or whenever a new task execution is started or completed at the server \cite{kliazovich2012greencloud}.\\
\noindent\tab \textit{Cost-aware}: Similarly, cost-aware is also an important parameter. It is related to the execution of tasks and the increase of bandwidth usage and the energy spent in during the migrations. Thus, a trade-off between QoS and cost has to be defined. Moreover, migration results in an increase usage of computing resources that are performing non-useful work (overhead). Therefore, a cost model referring to the quantification in monetary terms of the usage of infrastructure service providers' resources is important, once it allows to apply a pay-as-you go model.\\
\noindent\tab \textit{Application models}: This is an important feature in terms of QoS because it allows specifying the computational requirements for the application and a specific completion deadline.\\
\noindent\tab \textit{Communication model}: CloudSim can model network components, such as switches, but lacks fine-grained communication models of links and Network Interface Cards (NIC) causing VM migration and packet simulation to be network-unaware \cite{malik2017cloudnetsim++}. CloudNetSim++, on the other hand, supports a simulation model of real physical network characteristics such as network congestion, packet drops, bit error, and packet error rates. Moreover, GreenCloud allows communications based on TCP/IP protocol. It allows capturing the dynamics of widely used communication protocols such as IP, TCP, UDP, etc. Whenever a message needs to be transmitted between two simulated elements, it is fragmented into a number of packets bounded in size by network Maximum Transmission Unit (MTU). Then, while routed in the data center network, these packets become a subject to link errors or congestion-related losses in network switches \cite{kliazovich2012greencloud}.\\
\noindent\tab \textit{Migration support}: This policy allows applying data placement techiniques (i.e. application and workload migration) to benefit high QoS.\\
\noindent\tab \textit{Mobility/Location-aware}: As already explained, mobility/location-aware is quite an essential feature in fog computing. It allows maintaining (as much as possible) the end-to-end latency as both users and servers move. There are few simulators that support this feature. For instance, in cloud environments, CloudAnalyst \cite{wickremasinghe2010cloudanalyst} is a tool whose goal is to support the evaluation of social network applications, according to the geographic distribution of users and data centers. However, in fog environments, to the best of this work knowledge, there is no support for mobility of fog nodes. The only that provides mobility/location-aware that is currently available, is MyiFogSim. It is an extension of iFogSim to support users' mobility through migration of VMs between cloudlets \cite{lopes2017myifogsim}.\\
