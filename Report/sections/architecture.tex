%!TEX encoding = UTF-8 Unicode
\section{Description of the Project}
\label{sec:Architecture}
%- Como pensam abordar a tese tendo em conta o problema, as contribuicoes a que propoe a dar e a arquitectura definida.
%- Descricao de alto nivel da arquitectura do sistema: Arquitectura do SW explicando os principais componentes e as funcoes que executam.
%- Como procederam as escolhas das ferramentas, linguagens de programcao, ambientes de desenvolvimento, hardware
%- Como prevem desenvolver a arquitectura proposta: por onde vao comecar. Como vao integrar os componentes, que dificuldades esperam encontrar e que estrategias planeam para as ultrapassar.

To this work is considered the typical architecture of fog computing, Fig. \ref{fog_architecture}, where apart from the fixed cloudlets, there are some mobile fog nodes that can be used to support individual applications or to support the fixes cloudlets, providing offloading support ..... tal como o paper do bus

%neste relatório consideramos que as cloudlets estão nos APs?
%A arquitetura dos Buses é boa (APs com várias cloudlets etc....)

%cloudsim plus explica o proque da escolha dos simuladores

%The concept of mobile fog computing is similar to fog computing, in which both IoT and fog nodes are mobile components. Those are connected wirelessly (e.g., via WiFi or Bluetooth). The challenge with implementing fog computing in mobile environments lies in the underlying complexity of data placement management and decision-making to ensure the QoS to all users (i.e. ensure that all latency constraints of users' applications are met).\\

%\noindent\tab Although several studies were already done in order to provide mobile support for IoT devices, the purpose of this study is to support mobile fog computing, once fog nodes can be anything in the path that connects things to the cloud. This distributed middle tier, in the 3-tier architecture (things-fog-cloud), can use as fog nodes any physical device that has facilities or infrastructures that can provide resources and visualization capabilities. This, may include movable fog nodes, such as cars, buses, unmanned aerial vehicles (UAVs), etc. The importance of mobile fog nodes cannot be overlooked, once they may represent a way to offload fixed cloudlet tasks and thus improve fog features. In this field there are already some early efforts.\\

%Starting from available simulators a significant programming effort is required to obtain a simulation tool meeting the actual needs.
%Simply applying existing radio access-oriented MM (mobility management) schemes leads to poor performance mainly due to the co-provisioning of radio access and computing services of the MEC-enabled (mobile edge computing) BSs (base stations).
%More specifically, Fog might be specified in terms of functionality as Fog edge nodes (FENs), Fog server (FS), and Foglet, where FENs and FS are hardware nodes, and Foglet is the middleware in charge of data exchange, as presented in Fig. 1.


%\noindent\tab Fog computing will be crucial in a diversity of scenarios. For
%instance, heterogeneous sensory nodes (e.g., sensors, controllers, actuators)
%on a self-driving vehicle, are estimated to generate about 1GB data per second
%\cite{angelica2013google}. As the number of features grow, the data deluge
%grows out of control. Moreover, these types of systems, where people's lives
%depends on it, are hard real-time what means that it is absolutely imperative
%that all deadlines are met. Offloading tasks to fog nodes will be the best
%solution, once a big effort in mobility support has been done through the
%migration of VMs using cloudlets \cite{lopes2017myifogsim}. Also, in this
%context, Puliafito et al. address three types of applications where fog is
%required, namely, citizen's healthcare, drones for smart urban surveillance and
%tourists as time travellers \cite{puliafito2017fog}, addressing the needs of
%low latency and mobility support.\\
% another works were performd using MDP-based approaches (ex. Distributed Autonomous Virtual Resource Management in Datacenters Using Finite-Markov Decision Process)