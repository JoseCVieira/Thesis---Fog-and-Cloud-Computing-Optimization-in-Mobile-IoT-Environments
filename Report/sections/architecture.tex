%!TEX encoding = UTF-8 Unicode
\section{Description of the Project}
\label{sec:Architecture}
%- Como pensam abordar a tese tendo em conta o problema, as contribuicoes a que propoe a dar e a arquitectura definida.
%- Descricao de alto nivel da arquitectura do sistema: Arquitectura do SW explicando os principais componentes e as funcoes que executam.
%- Como procederam as escolhas das ferramentas, linguagens de programcao, ambientes de desenvolvimento, hardware
%- Como prevem desenvolver a arquitectura proposta: por onde vao comecar. Como vao integrar os componentes, que dificuldades esperam encontrar e que estrategias planeam para as ultrapassar.

To this work is considered the typical architecture of fog computing, Fig. \ref{fog_architecture}, where apart from the fixed cloudlets, there are some mobile fog nodes that can be used to support individual applications or to support the fixes cloudlets, providing offloading support ..... tal como o paper do bus

%neste relatório consideramos que as cloudlets estão nos APs?
%A arquitetura dos Buses é boa (APs com várias cloudlets etc....)

%cloudsim plus explica o proque da escolha dos simuladores

%The concept of mobile fog computing is similar to fog computing, in which both IoT and fog nodes are mobile components. Those are connected wirelessly (e.g., via WiFi or Bluetooth). The challenge with implementing fog computing in mobile environments lies in the underlying complexity of data placement management and decision-making to ensure the QoS to all users (i.e. ensure that all latency constraints of users' applications are met).\\

%\noindent\tab Although several studies were already done in order to provide mobile support for IoT devices, the purpose of this study is to support mobile fog computing, once fog nodes can be anything in the path that connects things to the cloud. This distributed middle tier, in the 3-tier architecture (things-fog-cloud), can use as fog nodes any physical device that has facilities or infrastructures that can provide resources and visualization capabilities. This, may include movable fog nodes, such as cars, buses, unmanned aerial vehicles (UAVs), etc. The importance of mobile fog nodes cannot be overlooked, once they may represent a way to offload fixed cloudlet tasks and thus improve fog features. In this field there are already some early efforts.\\